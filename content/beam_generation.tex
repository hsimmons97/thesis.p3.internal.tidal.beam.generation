\documentclass[12pt]{article}
\input{/home/dmitry/Work/Research/thesis/FINALE/settings.tex}
\graphicspath{{/home/dmitry/Work/Research/thesis/FINALE/P3_ITS_GENERATION/figures/}}

%\documentclass[PhD_Thesis.tex]{subfiles}

\begin{document}
	\iftoggle{only_Chapter} {
		\title{Generation of internal tidal beam in Tasman Sea}
		\maketitle
	}

\section*{Abstract}
Macquarie Ridge south of New Zealand is a strong generator of internal tidal waves (ITs) which 
later form a prominent internal tidal beam, a tight filament of baroclinic tidal energy. The 
mechanism of barotropic energy conversion into baroclinic field is investigated by means of 
numerical experiments with prescribed mesoscale conditions corresponding to different background 
conditions. The obtained energy transfer rates into IT mode-1 show variability both in its 
magnitude and spatial distribution. This variability is controlled by resonance between local 
barotropic forcing and remotely generated ITs. The remote baroclinic signal takes its origin at an 
easterly located Campbell Plateau. Their pathway is drastically different from the progressive 
plane waves due to interference with the ridge originating waves and consequent formation of a 
standing Poincare wave. This complex internal tide environment is modulated by the ratio of the 
respective East and West traveling components. \textit{The west part associated with the plateau 
experiences variability following the upper ocean stratification. (this currently questionable)} 
Under conditions of the Tasman Sea subtropical frontal zone located northward of the region, 
leading to weaker stratification, lower amplitude waves, a more developed standing wave and hence, 
less energy transfer into the internal tidal beam. Conversely, under different realization an 
intrusion of warm subtropical waters created conditions for stronger generation resulting in high 
across Macquarie ridge energy transport and larger conversion rates. This mechanism of variability 
is further shown to affect position of the internal tidal beam. It is characterized by complex 
pattern of nodes and antinodes. The simple inverse analytical models shows that spatially 
wide-spread generation is one to blame. Than it is shown that the beam wobbles as a result in 
shifting of internal tide production hotspots. This identifies a reason for beam long range 
nonstationarity as a result of generation nonstationarity.\\
The inverse model based on elliptical waves shows that the beam undergoes changes in its appearence similar to Fresnel and Fraunhoffer diffraction in optics.

\section{Introduction}
Baroclinic semidiurnal tides originate as a strong barotropic flow occurs over steep topography in stratified ocean. This process is though as a scattering of barotropic tidal energy into baroclinic motions (\cite{ROG:ROG324}). It is quite definitely known that the scattering has a large impact on energetics barotropic tide and is a part of energy dissipation (\cite{egbert2000significant}, \cite{munk1997once}) with 1/3 of $M_2$ energy being dissipated through generation of internal tide.\\
Analytical models of internal tide generation points to importance of ratio between direction of 
internal wave characteristics to bathymetry slope (\cite{garrett2007internal}). Omitting nonlinear 
regimes for low Froude number flows, highly supercritical slopes will give rise mostly to the 
gravest mode of internal tide (\cite{echeverri2010internal}). Such a case for Macquarie Ridge 
located south of New Zealand. Here barotropic tide energy is conversed into largely mode-1 
baroclinic tide.\\
Variability of internal tide generation can occur to number of reasons. Stratification could have a direct impact in WKB sense creating less prominent obstacle (\cite{holloway1999internal}). But this phenomena is more pronounced for bathymetry touching pycnocline. More recently, it became apparent that the generation is more unstable in presence of remote baroclinic signals. In the most well studied setting of two parallel ridges a resonance conditions occurs (\cite{echeverri2010internal}, \cite{buijsman2012double}, \cite{buijsman2014three}) that would lead either to intensification or destruction of energy transfers. More generally, phenomena of coupling at internal tide production sites of local generation and remote signal can lead to nonstationarity in wave generation (\cite{Kelly2010a}, \cite{osborne2011spatial}, \cite{kerry2013effects}, \cite{xing1998three}, \cite{buijsman2012modeling}). The notion of variability in generation was observed as well (\cite{pickering2015structure}, \cite{zilberman2011incoherent}). The same phenomena occurs at Macquarie Ridge where behind located Campbell Plateau sends internal tides which results in generation variability.\\
The ubiquitous feature of mode-1 internal tides as they leave production sites is their appeareance 
as tightly confined beams of energy propagating without much spatial spreading over large distances 
(\cite{zhao2016global}). This surprising feature is believed to be related to multiple generation 
sites (\cite{rainville2010interference}, \cite{terker2014observations}). Especially, this is true 
for long swaths of generation bathymetry where point source two dimensional models of generation 
becomes unrealistic (\cite{munroe2005topographic}).\\
The question of stationarity of the tidal beams is still open. The most obvious reason for 
variations is nonlinear interaction with variable mesoscale conditions in the open sea 
(\cite{kerry2016quantifying}, \cite{dunphy2014focusing}, \cite{chavanne2010surface}, 
\cite{ansong2017semidiurnal}, \cite{dunphy2017low}, \cite{zaron2014time}, 
\cite{doi:10.1080/03091927509365798}, \citep{kelly2016internal}, \citep{park2006internal}, 
\cite{rainville2006propagation}). On the first order, refraction by variable stratification and 
hence, propagation speed is a key processes or reflection at strong frontal zones. Oppositely, here 
we will provide analysis suggestive for a different point such that variable spatial conversion 
rates will lead to displacement of Tasman Sea tidal beam.\\
This is done by investigation of numerical experiments in Regional Ocean Modeling System with 
different realistic background conditions (Section 2). From derived energy characteristics it is 
shown that variations occur both at generation regions (Section 3a), but also in open ocean 
(Section 3b). Then developed analytical solution for spatially confined knife-edge barrier and 
semi-analytical solution (Section 3c) can describe the beam properties from pure generation 
picture. This provides rough estimates on generation-related nonstationarity. The work is concluded 
with inter-comparison between numerical experiments and available observational data (Section 4). 
In conclusion wide-spread application is provided (Section 5).\\

\newpage

\section{Numerical experiments and analysis}
\subsection{Numerical experiments}
To study variability of internal tide generation around New Zealand and later propagation numerical 
simulations were carried out with Regional Ocean Modeling System \citep{shchepetkin2005regional}. 
The numerical domain covered southern Tasman Sea from subantractic waters of $60^{\circ}$ S 
to subtropics in $35^{\circ}$ S. And the zonal extent stretched from $142^{\circ}$ to $172^{\circ}$ 
E. This ensued correct representation of reach regional, oceanographic conditions. The horizontal 
grid spacing was taken to be of $1/32^{\circ}$ corresponding on average to discretization of 3 km 
in zonal direction and 2.5 km in meridional. The nonuniformly separated, vertical 50 $s$-levels 
were placed to smoothly follow subsurface terrain.\\
Such discretization of vertical momentum equation tends to induce artificial, horizontal  
along-slope flows \citep{haidvogel1999numerical} due to errors in reproducing of pressure 
gradient force. Especially severe errors are made by steep terrain. The misbehavior is usually 
solved by aritificial smoothing of topography. Here this is done by spatial filtering 
\footnote{How it was done?}. This procedure has an adverse effect on internal tide generation  
\citep{di2006numerical} since production sites are found where bottom gradients are large. To 
test sensibility of numerical setup, a sensitivity study was carried out with simulations of  
$1/8^{\circ},~1/16^{\circ},~1/64^{\circ}$ horizontal resolution. The essential for this study 
internal tide behavior appeared at $1/16^{\circ}$ and converged for $1/32^{\circ}$ and 
$1/64^{\circ}$ cases where no marked difference was observed, except a substantial increase in 
high mode content which is in line with \footnote{previous investigations} 
\citep{di2006numerical}.\\
This work addresses the gravest baroclinic mode dynamics in the deep ocean. Spatial extent of 
waves is large compared to associated vertical displacements. This ensures linear regime of 
propagation without dispersive and nonhydrostatic effects taken place such as fission into 
solitons. A hydrostatic solver used in ROMS seems to be a proper choice for the simulations. Such 
simplification in wave dynamics was assumed in previous studies (\citep{carter2008energetics, 
merrifield2001generation,  merrifield2002model, kerry2013effects}). In the nonhydrostatic 
simulations \citep{kang2012energetics, zhang2011three} the nonhydrostatic effects are found to be 
important for shoaling internal tides, while for main part generation follows linear dynamics with 
vertical accelerations to have a negligible contribution.\\
\textbf{Frictional forces}\\
The horizontal boundary conditions were imposed to be open for depth-averaged, barotropic flows 
following \footnote{Shchepetkin's condition, is there any reference?}. The baroclinic fields are 
nudged to zero by linear increased lateral viscosity and diffusivity over sponge layers. Through 
the same outer boundaries numerical simulations were forced with barotropic tide. The tidal 
currents and sea level are derived from TPXO atlas, version 7.2 \citep{egbert2002efficient} and 
prescibed as linearly interpolated volume transports. It was used only the largest semidiurnal 
constituent $M_2$. The amplitude ratio between the principal lunar and solar components are 4-to-1 
suggestive of slight open-ocean spring-neap modulation. The diurnal species are weak in the region 
except shoals east of New Zealand \citep{walters2001ocean}\footnote{Do I need to justify M2 only 
choice?}.\\
To investigate variations of baroclinic tide dynamics several ocean states were prescribed and 
analyzed separately. In the simplest setting, no lateral gradients in water density were given and 
vertical gradient was representative of Tasman Sea basin climatological mean. The second set of 
simulations was comprised to investigate interannual and interseasonal variability 
(Table 1). And the third calculation was intended to provide context for TTIDE field program, so 
this was done by performing a single experiment once initialized and then let run for three 
numerical months.
\begin{table}
	\caption{Carried out numerical experiments}
	\begin{tabular}{ |p{3cm}||p{5cm}|p{5cm}|  }
		\hline
		\multicolumn{3}{|c|}{Numerical experiments used in this study} \\
		\hline
		Experiment abbreviation & Dates & Comments (reason?) \\
		\hline
		Uniform & ~ & No mesoscale \\
		2012 &   Jan 1st - Jan 15th, 2012 & Interannual \\
		2013 &   Jan 1st - Jan 15th, 2013 & Interannual \\
		2014 &   Jan 1st - Jan 15th, 2014 & Interannual \\
		2015 &   Jan 1st - Jan 15th, 2015 & Interannual \\
		2013\_274 &   Oct 1st - Oct 15th, 2013 & Interseaonal \\
		2015\_074 &   Mat 1st - Mar 15th, 2015 & Interseaonal \\
		2015\_TTIDE &   Jan 1st - Mar 1st, 2015 & Field period \\
		\hline
	\end{tabular}
	\label{ch2:table_exp}
\end{table}
The simulations with variable conditions were at first initialized by HYCOM hindcasts 
\footnote{cite?} for respective start date. Then during temporal integration, along with 
barotropic tidal flow, time-variable, subtidal two dimensional fields \footnote{(vertical 
coordinate and along boundary coordinate)} of horizontal currents, temperature and salinity were 
prescribed into the numerical ocean. The air-sea interaction was also given from MERRA-reanalysis 
\citep{rienecker2011merra} by insolation, air  temperature, EP rates and most imporantly, wind 
stresses.

\subsection{Internal tide analysis}
As it is seen in table 1, the simulations were carried for 15 days or longer. The first 10 days 
were given for spin up of baroclinic tide generation and propagation. Roughly, it takes about 7 
days for the mode-1 signal to cross Tasman Sea basin from New Zealanda to Tasmania. After that 
period, three dimensional fields velocity, temperature and salinity were sampled hourly. These were 
later subject to high pass filtering. Butterworth filter of order $x$ with cut off time of $30$ 
hours were applied to the records. This produced tidal signal that was further fit in a least 
square sense to the principle semidiurnal harmonic. Then the three dimensional fields was then 
underwent a routine separation of barotropic and baroclinic signals \citep{cummins1997simulation, 
kunze2002internal, carter2008energetics}, depth-averaged current is thought to represent a pure 
barotropic signal and any vertical deviation is attributed to a baroclinic wave,
\begin{equation}
\label{ch2:bt_bc_vel}
\vec{u}_{bt}(x,y) = \frac{1}{H} \int_{-H}^{0} \vec{u}(x,y,z)  dz,~\vec{u}_{bc}(x,y,z) =  
\vec{u}(x,y,z) - \vec{u}_{bt}(x,y)
\end{equation}
To describe distribution of pressure at first, from a linear equation of state and respective 
TS-fields density perturbation from the reference is obtained. Then the hydrostatic approximation 
is employed and after vertical integration the total pressure fields is found. This is then subject 
to barotropic condition, so that baroclinic field is taken to be as a deviation from the 
depth-averaged,
\begin{equation}
\label{ch2:bt_bc_pres}
p(x,y,z) = \int_{-z}^{0} \rho(x,y,z) dz,~p_{bc}(x,y) = p(x,y,z) - \frac{1}{H} \int_{-H(x,y)}^{0} 
\rho(x,y,z) dz
\end{equation}
In the both expressions rigid-lid approximation is used. This is a valid statement unless vertical 
accelerations are smaller than acceleration due to gravity which is true except shallow depths 
\cite{kelly2010}.\\
Each dynamical variable was then decomposed into vertical modes. The structure functions were 
obtained by using local Brunt-Vaisala frequency profiles found from averaged density fields. These 
were used in Sturm-Liouville problem for the hydrostatic approximation,
\begin{equation}
\frac{d}{dz}(\frac{\omega^2 - f^2}{N^2} ) \frac{d \psi(z)}{dz}) + c^2_n \psi(z) = 0
\end{equation}
where $c_n$ is a mode respective phase speed in nonrotating ocean. The first 3 vertical modes were 
fit three-dimensional fields. And only mode-1 was used in the Chapter.\\
Now energy diagnostics could be obtained. First, depth-averaged mode-1 energy flux will be simply,
\begin{equation}
\vec{F} = \frac{1}{2} \frac{1}{H} \cj{\vec{u}} p \int_{-H}^{0} \psi_1(z) \psi_1(z) dz
\end{equation}
At second, rates of conversion from barotropic to baroclinic \citep{simmons2004internal, 
kurapov2003m} were calculated as,
\begin{equation}
C_{bt\to 1} = -\frac{1}{2}(\cj{\vec{u}_{bt}} \cdot \nabla H) p_{1}
\end{equation}
The fraction $\frac{1}{2}$ in front of the energy characteristics appear because harmonic, complex 
amplitudes are used in the expressions.\\
These calculations had produced a set of dynamical variables related to barotropic and baroclinic 
fields in each experiment (table.\ref{ch2:table_exp}). The obtained values hereafter will be 
referred as a realization. For instance, the longest experiment, 2015\_TTIDE had 10 realizations 
which are all dependent since it was the continuous simulation. The latter calculation of 
stationarity was made by simple expressions,

\subsection{Discrete Fourier Decomposition by inverse modeling}
In addition to the above characteristics the mode-1 internal tide field was subject to directional 
analysis in order to remove interference modulations by a novel approach. Though similar methods 
were used previously in internal tide field programs \citep{hendry1977observations, 
lozovatsky2003spatial} that were based on array beamforming method and stationarity of the field or 
satellite altimetry \citep{dushaw2002mapping} or in surface wave studies 
\citep{longuet1961observations, munk1963directional, long1986inverse}. Let 
mode-1 pressure in complicated seas to be described by an angular spectrum
\begin{equation}
\label{C1:eq.spectrum}
p(\vec{r}, t) = \int_0^{2\pi}  d\theta_k S(\theta_k) e^{i \vec{k}(\theta_k) \cdot \vec{r} + 
\phi(\theta_k) - i \omega t}
\end{equation}
Here each elementary (monochromatic) sine wave of wavenumber $k$ travels in direction $\theta$ with 
energy $S(\theta)^2 d\theta$ and temporal (spatial) lag of $\phi(\theta)$. The statement can be 
reformulated in terms of Fourier coefficients \citep{munk1963directional} by application of 
Jacobi-Anger expansion,
\begin{equation}
p(r, \theta) = e^{i \vec{k}(\theta) \cdot \vec{r}} = \sum_{m = -\infty}^{m = \infty} i^{m} J_{m}(k 
r) e^{im(\theta - \theta_k)}
\end{equation}
shows that a field at point $(r, \theta)$ produced by plane wave can be expanded in series of 
Bessel functions and circular functions. Then its substitution into \eqref{C1:eq.spectrum} and 
reorganization lead to
\begin{equation}
\label{C1:eq.series}
p(r, \theta) = \sum_{m=-\infty}^{m=\infty} \big[ \int_0^{2\pi}  d\theta_k S(\theta_k) 
e^{i\phi(\theta_k)} e^{-im\theta_k} \big] i^m J_m(kr) e^{im\theta}
\end{equation}
Term in brackets (square brackets) represent convolution integrals defining Fourier coefficients of 
order $m,~A_m - i B_m$. Thence, series \eqref{C1:eq.series} state a model equation to find the 
unknown coefficients from the known, measured pressure field that were sampled at a set of points 
$(r_i, \theta_i)$ and if infinite series is truncated at some order $N$. Real and imaginary parts 
will constitute two separate problems allowing deterministic definition of the spectrum.\\
The same steps are repeated but with current velocities instead. By invoking plane wave  
polarization relations \citep[e.g.,][]{muller2000scattering} are inserted into 
\eqref{C1:eq.spectrum} and the following equations are found,
\begin{align}
\label{C1:uv.eq}
\begin{Bmatrix}
u_i \\ v_i
\end{Bmatrix}
= \frac{1}{2} \sum_{m = -N}^{m = N} J_{m} (kr_i) e^{im(\theta + \pi/2)}
\begin{Bmatrix}
(\omega - f) A_{m + 1} + (\omega + f) A_{m - 1} - i [(\omega - f) B_{m + 1} + (\omega + f) B_{m - 
1}] \\ 
(\omega - f) B_{m + 1} - (\omega + f) B_{m - 1} + i [ (\omega - f) A_{m + 1} - (\omega + f) A_{m - 
1}]
\end{Bmatrix}
\end{align}
The dependence of currents on wave bearing causes splitting of Fourier coefficients and 
asymmetry via Coriolis effect. This results points out that to describe velocity field 
higher circular harmonics have to be used. Physically, velocity field has higher spatial 
wavenumber. But in \eqref{C1:uv.eq} additionally, the asymmetry is observed 
for clockwise and counterclockwise components.\\
An inverse model combines dynamical relations of \eqref{C1:p.eq} and \eqref{C1:uv.eq} into a matrix 
equation
\begin{equation}
y = K x
\end{equation}
Generally, it is unstable to small errors in data and produce physically inconsistent results. This 
can circumvented by seeking a damped least square solution \citep{munk2009ocean} where a 
minimization function is given by
\begin{equation}
\label{C1:Tikh_prob}
J = ||K x - y||^2_2 + \alpha ||x||^2_2
\end{equation}
The unknown regularization parameters $\alpha$ acts as a high-pass filter in a singular value 
decomposition of $K$ \citep{bennett1992inverse}. In field studies this is usually set by a 
signal-to-noise ratio (\cite{munk2009ocean}), since the parameter scales noise variance (residue) 
to actual signal's strength. To obtain $\alpha$ in data-driven way a straightforward approach is 
adapted that based on 
trade-off curve method (\cite{hansen1993use}). In \eqref{C1:Tikh_prob} amount of allowed error 
is competing with solution's variance. An optimal parameter should balance these factors. This is 
seen as a rapid change in behavior of curve associating residue with model's norm as regularization 
varie. In most cases the curve has a sharp corner connecting aforementioned limits, hence, the 
method's name is a L-curve (\cite{hansen1999curve}). And the corner is to occur for an optimal 
regularization parameter.\\
The equations \eqref{C1:eq.series} and \eqref{C1:uv.eq} are sampled at locations in a concentric 
arrays placed at $\lambda,~0.5\lambda,~0.25\lambda$ where $\lambda$ is a local mode-1 wavelength. 
At each location $u,~v,~p$ are used as data and for a region embraced by array Fourier coefficients 
are found. And these then are used in reconstructions.\\
The method used here is different from \citep{zhao2010long} for two main reasons. The model 
equations produce simultaneous fit of all the components, rather than a finite number of a single 
directed plane waves. This can make a difference in regions where diffraction is important such as 
near internal tide generation or scattering regions. And at second, velocity field is utilized 
which provides an additional constrain. Moreover, in synthetic experiments with 
\eqref{C1:Tikh_prob} where instead of $L2$-norm regularization it was used $L1$-norm, the results 
were approaching one of plane wave technique of \citep{zhao2010long}. Additionally, the proposed 
method can be utilized for a single mooring where half-space separation is necessary.

\section{Results}
\subsection{Ridge}
\subsection{Mid-basin beam}
The mid basin properties of the tidal beam include:
\begin{itemize}
	\item wobbling
	\item slightly increased phase speed
	\item large deviations of group speed
	\item wavefronts
	\item spotty APE and HKE
\end{itemize}
\begin{figure}
	\mfig[0.5]{slide_3.pdf}
	\mfig[0.5]{slide_4.pdf}
\end{figure}
These properties point to use that the beam is a standing wave. But no apparent reflected wave is 
seen. Generation at Southern Tasman Plateau and Cascade Seamount can not be a source for opposite 
traveling waves due to their weakness. Additionally, directional spectra or plane wave fit does not 
show presence of such waves. But here we will show that generation indeed is reponsible for wobbly 
pattern of the tidal beam and present a method that connects conversion pattern along Macquarie 
Ridge with mid basin characteristics of the beam.


Study the wavefield varies depending on size of the ridge.
\begin{figure}
	\mfig[0.5]{res_test.png}
	\caption{Comparison of inverse solution and numerically diagnosed.}
\end{figure}
\subsection{Analytical results}

\section{Discussion}
\subsection{Generation}
Uniform: Three beams similar to satelite. Central primarily originates in main Macquarie Ridge. 
There is large  variability in the northern, southern, but central is stable.\\
There two major generation hotspots associated with steep submarine seamounts (Figure 1).\\
\begin{figure}
	\mfig[0.5]{uni_flux.png}
	\mfig[0.5]{Clin_Mac.png}
	\caption{Energy flux map of the beam in Tasman Sea (a) and conversion rates at Macquarie Ridge 
	(b).}
\end{figure}
The other region of generation is behind located Campbell Plateau.\\
Additionally, there is a standing wave.
\begin{figure}
	\mfig[0.5]{stand_wave_2by2.png}
	\caption{Standing wave}
\end{figure}
Now we investigate interaction between conversion and standing wave.\\
Spatial maps of conversion rates show variation in magnitude and location of conversion hotspots 
(Fig. 1). The energy transfers predominantly occur along steep flunks of two seamountains. The most 
striking difference was found between '2014' and 'uniform' experiment.\\

Conversion rates vary between experiments. Knife edge calculations with WKB-scaled depth does not 
confirm the same variations (Figure 2).\\
The other possibility for variations can be seen due to presence of remote waves.
Model conversion rates and knife edge, Enhancement of generation due to remote propagating tides.\\
\begin{figure}
	\mfig[0.5]{knife_edge.png}
	\caption{Variation of integrated conversion rates and knife edge}
\end{figure}

\begin{figure}
	\mfig[0.25]{cmp_14_15_params_MR1_flood_tf.png}
	\mfig[0.25]{cmp_14_15_params_MR1_max_tf.png}
	\mfig[0.5]{years_clins.png}
	\caption{Model output}
\end{figure}

So variations in standing wave leads to variations in conversion.\\

\subsection{Variation of standing wave, HYCOM}
We represent standing wave as interference of  two waves (\cite{martini2007diagnosing}),
....\\
\begin{figure}
	\mfig[0.5]{1.pdf}
\end{figure}

\begin{figure}
	\mfig[0.5]{HKE_44.png}
\end{figure}
Discussion on variability of standing wave: stratification?\\
It was tested hypothesis that the stratification controls, but results are inconclusive.\\

\section{Discussion: Field observations and Boundary condition for scattering}

\section{Conclusions}

\newpage
\iftoggle{only_Chapter} {
	\appendix
}

\nottoggle{only_Chapter} {
	\addcontentsline{toc}{section}{Appendices}
}

\section*{Appendices}

\renewcommand{\thesubsection}{\Alph{subsection}}
\setcounter{subsection}{0}
\subsection{Analytical model for knife edge}
Let consider a simplified problem of the internal tide generation at a three dimensional ridge. That is generating topography has extent in along x-axis, $a$, but infinitely small width. We will not pursue full solution of the problem and will not seek actual amplitudes of baroclinic modes, but rather concentrate on defining spatial pattern of the generated waves (might move up to opening). Under such problem statement the actual height is of no importance. And hence, in frame reference traveling with barotropic current, the topography becomes a piston-alike wavemaker that sends out the lowest mode internal tide. Its behavior in flat bottom ocean is well represented by Laplace tidal equations (cf \cite{kelly2012cascade}):
\begin{align}
\vec{u}_t + 2 \Omega \vec{k} \times \vec{u} = - \frac{1}{\rho_0} \nabla \cdot p\\
\nabla \cdot \vec{u} = -(\frac{1}{N^2 - \omega^2}p_z)_{tz}
\end{align}
with boundary conditions on the ridge,
\begin{equation}
\vec{u}\cdot \vec{n}|_{ridge} = \vec{u}_{bt} \cdot \vec{n} |_{ridge}
\end{equation}
And boundary condition in the infinity implying outgoing waves. Note that barotropic current, in general, is given as a current ellipse written in complex form as $\vec{u}_{bt} = u_{bt} + i v_{bt}$. This means that the boundary condition does not have a simple harmonic form. But the barotropic current can be decomposed to clockwise (CW) and counterclockwise components,
\begin{equation}
\vec{u}_{bt} = W_{CW} e^{-(i \omega t - \phi_{CW})} + W_{CCW} e^{(i \omega t + \phi_{CCW})}
\end{equation}
and the generation of internal tide can be solved separately for two oppositely rotating currents. In further discussion the only one component (CCW) is taken care of since a solution will have similar form with differences arising in sign in front of tidal frequency.\\
Considering rigid lid approximation and impermeable bottom equations for eigen modes is solved such that wavelength of corresponding mode is introduced. Than the dynamical equations can be reformulated as Helmholtz equation with harmonic temporal dependence implied and appropriate polarization relations,
\begin{align}
\nabla^2 p + k_n^2 p = 0\\
u = \frac{-i \omega p_x + f p_y}{\omega^2 - f^2},~v = \frac{-i \omega p_y - f p_x}{\omega^2 - f^2}
\end{align}
Since the problem is now formulated in terms of pressure only, boundary condition takes the following form (\cite{greenspan1968theory}),
\begin{equation}
\Big[ \frac{-i \omega p_y - f p_x}{\omega^2 - f^2} \Big]_{ridge} = \vec{u}_{BT}|_{ridge}
\end{equation}
The above equations (5-8) state generation of internal tides by vibrating strip of width $a$. Such problem was solved for radiation of acoustic waves by \cite{morse1946methods}. Since in Cartesian coordinates the strip boundary condition does not allow separation of variables, one can employ elliptic coordinate system,
\begin{equation*}
x = \frac{a}{2} \cosh \mu \cos \theta,~y = \frac{a}{2} \sinh \mu \sin \theta
\end{equation*}
where edges of the strip will be focii of ellipse, boundary condition will take simpler form,
\begin{align}
\Big[ \frac{-i \omega p_{\mu} + f p_{\theta}}{\omega^2 - f^2} \Big]_{\mu = 0} = \frac{a}{2} \sin \theta (\vec{u}_{BT}|_{\mu = 0})
\end{align}
The factor $\frac{a}{2} \sin \theta$ arises via conversion from Cartesian to elliptical derivatives. Note change of sign for $p_{theta}$ because opposite growth of $\theta$ argument to $x$ argument. Laplace operator in the elliptical coordinates has eigensolutions (''sloshing modes") in form of Mathieu functions (\cite{stratton2007electromagnetic}) that solution for the emitted waves will have form of,
\begin{equation}
p \sim \sum_{j} [Se_j, So_j](h, \theta) [Je_j, Jo_j, Ye_j, Yo_j](h, \mu)
\end{equation}
with $Se_j,~So_j$ - angular Mathieu functions of order $j$ corresponding to $\cos$ and $\sin$, $Je_j,~Jo_j$ and $Ye_j,~Yo_j$ - radial Mathieu functions of the first and second kind of order $j$ corresponding to Bessel functions. The physical parameter that sets Mathieu functions behavior is $h = {a k_n}{4} $ so that in large distance limit, $h \mu \gg 1,~rk_n \gg a$, they will represent circular harmonics.\\
Now RHS of boundary condition (9) can be expressed in series of Mathieu functions,
\begin{equation}
\sin \theta = \sum_{j = 0}^{\infty} C_{2j + 1} So_{2j + 1} (\theta)
\end{equation}
with coefficients $C_{2j + 1}$ defined by normalization constants $N_{2j + 1} = \int_{0}^{2\pi} So_{2j + 1}^2(\theta) d \theta$ and the first Fourier coefficient of $So_{2j + 1}$. The above series for $\sin$ and form of the boundary condition suggest solution in the form,
\begin{equation}
p(\mu, \theta) = \sum_{j = 0}^{\infty} \big( A_{2j+1} So_{2j + 1}(\theta) + B_{2j+1} Se_{2j + 1}(\theta) \big) Ho^{1}_{2j + 1}(\mu)
\end{equation}
Here $Ho^{1}_{2j + 1}(\mu) = Jo_{2j + 1} + i Yo_{2j + 1}$ is a Hankel-Mathieu function. For CCW component to ensure condition for outgoing radiation, the sign should be different, so $Ho^{2}_{2j + 1}(\mu) = Jo_{2j + 1} - i Yo_{2j + 1}$.\\
Now substituting (11, 12) into (9) the unknown coefficients $A,~B$ are obtained,
\begin{align*}
(-i\omega (\sum A So + B Se) Ho^{\prime} + f (\sum A So^{\prime} + B Se^{\prime}) Ho) = \sum (\omega^2 - f^2) C So
\end{align*}
At first, multiplying above equation by $So_{2m + 1}$ and taking integral from 0 to $2 \pi$, so that $\int_{0}^{2\pi} So_{2m + 1} Se_{2j + 1} d \theta = \int_0^{2 \pi} So_{2j + 1}^{\prime} So_{2m + 1} d \theta = 0,~\int_0^{2 \pi} So_{2m + 1} So_{2m+1} d \theta = No_{2m + 1},~\int_0^{2 \pi} So_{2m + 1} Se_{2j + 1}^{\prime} d \theta = -{N^{\prime}}^{2m+1}_{2j + 1}$. In the last statement orthogonality between $So$ and $Se^{\prime}$ is not satisfied. Equation on each order is obtained
\begin{align}
(-i \omega A_{2m + 1} No_{2m + 1}^o Ho_{2m + 1}^{\prime} + f \sum_{j} B_{2j + 1} {Neo^{\prime}}^{2m + 1}_{2j + 1} Ho_{2j + 1}) = C No_{2m + 1}
\end{align}
And at second, carrying out the same procedure but with $Se_{2j + 1}$,
\begin{align}
(-i \omega B_{2m + 1} Ne_{2m + 1} Ho_{2m + 1}^{\prime} + f \sum_j A_{2j + 1} {Noe^{\prime}}^{2m + 1}_{2j + 1} Ho_{2j + 1}) = 0
\end{align}
Note that $\int_0^{2\pi} Se^{\prime}_{2j + 1} So_{2m + 1} d \theta = -\int_0^{2\pi} So^{\prime}_{2j + 1} Se_{2m + 1} d \theta$, i.e. ${Noe^{\prime}}^{2m + 1}_{2j + 1} = -({Neo^{\prime}}^{2m + 1}_{2j + 1})^T$.\\
Equations (13) and (14) form a linear system to find coefficients for different component of the total field. These equations are solved numerically with $j_{max} = 5$ due to rapid convergence of the involved series.

\subsection{Inverse model}
The model closely follows ideas used in ref-to-Luc, 2010 and -Jody-2016. The internal tide generating ridge is given by point sources each emitting following
\begin{equation} \label{invm_eq:1}
p = p_{0} \frac{2}{\pi k d} \cdot e^{i  k  d}
\end{equation}
where $k$ - wavenumber associated with eigen mode-1, i.e. $k = \sqrt{\omega ^ 2 - f ^ 2}{c_{eigen} ^ 2}$, $d$ - distance between a point source and an observation point. By observation points here and after is meant points in which observations are inverted. The given solution is a solution of pressure distrubance propagation for two dimensional wave equation (p. 22, Frisk) and describes outgoing cylindrical wave. This is a far field approximation ($kd \ll 1$), in the near source zone the solution is substituded by Hankel functions. Here representation is simplified and observation points on the distance less than wavelength are omitted. Though introduction of Hankel function into the inverse model does not involve any additional complexity. By pressure here is thought mode-1 pressure amplitude that can be connected to sea level disturbance or isopycnal displacements.\\
To describe energy fluxes in the observational points polarization relations for cylindrical Poincare wave are invoked,
\begin{align} \label{invm_eq:2}
u = \frac{p_{0}}{\rho_{const}} * \frac{-i \omega \cos(\theta) + f \sin(\theta)}{\omega ^ 2 - f ^ 2} \cdot p_{\vec{d}}\\
v = \frac{p_{0}}{\rho_{const}} * \frac{-i \omega \sin(\theta) - f \cos(\theta)}{\omega ^ 2 - f ^ 2} \cdot p_{\vec{d}}
\end{align}
where $p_{\vec{d}}$ is a derivative along radius-vector $\vec{d}$,
\begin{equation}
p_{\vec{d}} = (i \cdot k - \frac{1}{2 d}) p
\end{equation}
In further description of the inverse model it is used following notation, indices $i,~k$ define $i,~k$-th point sources, while $j$ - $j$-th observation point.\\
The tidally and depth averaged energy fluxes will be given as an interference of pointwise fields from all sources,
\begin{align}
F_{j}^x = \frac{1}{2} \sum_k u_{kj}^{\star} \sum_i p_{ij} \int_H^0 \psi_1(z)^2 dz\\
F_{j}^y = \frac{1}{2} \sum_k v_{kj}^{\star} \sum_i p_{ij} \int_H^0 \psi_1(z)^2 dz
\end{align}
Note different indexes for u/v and p meaning that cross multiplication is involved which leads to complex interference pattern. In energy flux formulation normalization coefficient associated with eigenmode structure function are introduced by corresponding mode-1 structure function, $\psi_1(z)$. Coefficient $1/2$ is used for convenience to convert actual time averaging involved to multiplication of complex numbers. In further description the constant coefficients are omitted due to their irreleveance. The previous relations can be expressed in matrix form (it is not fully correct for fluxes, multiplication is done term by term per point),
\begin{align}
p_j = B^p_{ji}{p_i},~u_j = B^u_{ji}{p_i},~v_j = B^v_{ji}{p_i}\\
F^x_j = (B^u_{jk}{p_k})^{\star} B^p_{ji}{p_i},~F^y_j = (B^v_{jk}{p_k})^{\star} B^p_{ji}{p_i} \label{invm_eq:4}
\end{align}
where tensor notation is used, i.e. summation is done over same indices. Matrices $B^p_{ij},~B^u_{ij},~B^v_{ij}$ are short notation for generaion model and polarization relations, for example,
\begin{equation}
B^p_{ji} = p_i \frac{2}{\pi k d_j} \cdot e^{i  k  d_j}
\end{equation}
These can be thought as disretization of operators transforming distribution of sources into interference pattern in pressure and velocity fields.\\
Apparently, the energy flux relations are non-linear. To deal with this it is proposed an iterative technique. Let at $m$-th iteration there is a known distribution of wave amplitude at sources, $p^m_i$, the total energy flux field can be reconstruced by (\ref{invm_eq:4}). Than it is desired to find a small adjustment $\delta p^m_i$ (``nudge factor") such that residual between observed field and analytical description will be decreased. One can write,
\begin{align}
F_{j}^x = (B^u_{jk}(p^m_k + \delta p^m_k))^{\star} B^p_{ji}(p^m_i + \delta p^m_i) = \nonumber\\
(B^u_{jk} p^m_k)^{\star} B^p_{ji}{p^m_i} + (B^u_{jk} \delta p^m_k)^{\star} B^p_{ji} p^m_i + (B^u_{jk} p^m_k)^{\star} B^p_{ji} \delta p^m_i + (B^u_{jk} \delta p^m_k)^{\star} B^p_{ji} \delta p^m_i\nonumber\\
F_{j}^x - (B^u_{jk} p^m_k)^{\star} B^p_{ji}{p^m_i} = (B^u_{jk} \delta p^m_k)^{\star} B^p_{ji} p^m_i + (B^u_{jk} p^m_k)^{\star} B^p_{ji} \delta p^m_i + (B^u_{jk} \delta p^m_k)^{\star} B^p_{ji} \delta p^m_i \label{invm_eq:3}
\end{align}
The left hand side of (\ref{invm_eq:3}) represents the residual, the right hand side sets a controlling equation to obtain adjustment neceassary to decrease the residual. The last term of RHS shows a non-linear nature of the problem. This is omitted since the purpose of conseqeunt iterative technique is to find the final source distribution such that the model equations (\ref{invm_eq:4}) are satisfied in least square sense. Than the ``nudge-factor" can be found as inverse of 
\begin{equation}
F_{j}^x - (B^u_{jk} p^m_k)^{\star} B^p_{ji}{p^m_i} = R_j^x = \Big[ (B^u_{jk} )^{\star} B^p_{ji} p^m_i + (B^u_{jk} p^m_k)^{\star} B^p_{ji} \Big] \delta p^m_i \label{invm_eq:5}
\end{equation}
(these equations are not in matrix form, but obsevation point by observation point).
Hence, the aim of inverse model is to decrease error in representation of energy fluxes. The equation (\ref{invm_eq:5}) can be solved separately for zonal and meridional fluxes and also simultaneously for both directions. That is at each iteration step the nudge-factor is found first for zonal, than for meridional direction and finally, for both simultaneously. At the end pressure distribution is changed by average from all three substeps.\\
Note the inverse model equation (\ref{invm_eq:5}) is supported by additional condition stating that 
amplitude is nonnegative, $p_i^m + \delta p_i^m \geq 0$. All of this numerically is solved by 
linear programming routine \text{lsei} (least square with inequality) provided by LINPACK package.\\
Here it will be presented a test convergence and number tests on robustness on proposed iterative 
inverse model. The initial flux field is given by Fig. \ref{invm_fig:1} where by crosses are shown 
observational points. This define prescribed $F_{j}^x$ or $F_{j}^y$. Note that the prescribed field 
aims to describe midbasin energy flux field with Tasman shelf ommitted due to presence of 
reflection and complex bathymetry. The point sources distribution are given by green dots and at 
the first iteration step are set to $p^0_i = 100 Pa$. The distribution of points sources is 
representative to distribution of steep bathymetry which is belived to be an internal tide 
generator. In the inverse model there are only two parameters that describe characteristic of the 
internal tide, wavenumber and normalization coefficients used in energy flux. Both are found from 
solving eigenvalue for randomly picked stratification profile. This result in wavelength of 
$180~km$ which is a representative value for Tasman Sea conditions. In the same way eigenfunctions 
are obtained and normalization coefficients are found.\\
Hence, the inverse model does not account for
\begin{enumerate}
\item Bathymetry variation
\item Stratification variability
\item Variation of barotropic tide along ridges
\end{enumerate}
The first two points are thought to have minor effect on internal tidal beam structure. While the third is omitted to preserve simplicity of generation model. Additional tests were done with variation in barotropic tide phase along ridges, but they did not bring any substantial changes in foregoing results.\\
To show convergence of the inverse model it is given change in pressure amplitude with each iteration. Here convergence is defined by
\begin{equation*}
Conv = \sum_i \frac{(p_i^m - p_i^{m-1})^2}{(0.5 \cdot (p_i^m + p_i^{m-1}))^2}
\end{equation*}
The iterative solver is stopped when convergence is reaching tolerance. Here it is set to 0.01. From Figure (2a) it is seen that by 17th iteration there is no appreciable change in the inverse solution. This means that influence of non-linear terms in (\ref{invm_eq:3}) became negligible and the distribution of amplitude along the source region is the best in least square sense. The error of such description is given on subsequent panels of Fig. 2, where root-mean-square-error for different energy flux parameters is defined for example zonal component as
\begin{equation}
E_{x} = \sqrt{\frac{\sum_{obs} (F_i^x - \hat{F}_i^x)^2}{N_{obs}}}
\end{equation}
As it is seen the error is approaching stability for all used components much faster than 
convergence in amplitude. Note that the error is larger in zonal fluxes. The inverse solution can 
not predict far field behavior which is believed due to interaction with East Tasman Plateau. The 
obtained solution is given by Fig. 3. Here it is found that the inverse solution can not well 
represent the beam close to East Tasman Plateau. The following reasons can be named: interaction 
with topography and inadequacy of cylindrical wave model in the far-far field. It is believed that 
the second reason is the main. In general, the inverse solution picks up the central beam pretty 
well, outlines its boundary and the major region is satisfying manner. As well note that the 
northern and southern beams are also found in the solution.

\newpage
\section{Figures}

\begin{figure}
\end{figure}
\begin{figure}
\centering
\includegraphics[scale=0.75]{/home/dmitry/Work/Research/thesis/FINALE/P3_ITS_GENERATION/figures/aux/main_ridge_fl_modes.png}
\caption{Change in energetics of the beam.}
\end{figure}

\begin{figure}
\centering
\mfig[0.75]{converg.png}
\caption{Convergence}
\end{figure}

\begin{figure}
\centering
\mfig[0.5]{conv_fluxes.png}
\caption{Left panel - Overlaid isolines of energy flux magnitude of magnitude 1, 3, 5 kW/m. Middle panel - input data. Right panel - inverse solution.}
\end{figure}


\newpage{TO DO LIST}
\begin{itemize}
\item Polish: Clin and energy budget, knife edge, WKB-stretching, standing wave
\item Inverse model, elliptic waves are the way to go, why the phase can not be planar over much smaller stripes, something wrong with formulation of generation problem. See acoustics paper.
\item Why in new experiments the beam did not move?
\item Reasons for incoherence?
\item Is there going to be any seasonal cycle?
\end{itemize}

\bibliographystyle{apa}
\bibliography{/home/dmitry/Bibtex_lib/my_first_lib}

\end{document}