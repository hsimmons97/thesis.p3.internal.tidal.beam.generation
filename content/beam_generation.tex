\documentclass[12pt]{article}
\input{/home/dmitry/Work/Research/thesis/FINALE/settings.tex}
\doublespacing
%\graphicspath{{/home/dmitry/Work/Research/thesis/FINALE/P3_ITS_GENERATION/figures/}}

%\documentclass[PhD_Thesis.tex]{subfiles}

\begin{document}
	\iftoggle{only_Chapter} {
		\title{Generation of internal tidal beam in Tasman Sea}
		\maketitle
	}

\section*{Abstract}
Macquarie Ridge south of New Zealand is a moderate generator of internal tidal waves (ITs) forming 
a tidal beam. The beam was subject to study of TTIDE/TBEAM field program. Here, beam's generation 
and characteristics are investigated by means of numerical experiments with prescribed different 
mesoscale conditions. At the strongest site, conversion of barotropic tidal produce on average 
1.6 GW of baroclinic mode-1 and with a range of $30~\%$. This variation is mainly associated with 
amplitude of remote ITs originating on slopes of Campbell Plateau. The two almost parallel 
conversion sites form a system similar to a semi-enclosed resonator. Its efficiency of energy 
extraction is shown to depend on local stratification by (a) changing WKB-scaled topography and (b) 
phase lag between the sites. The produced baroclinic waves are radiated into deep ocean as a 
spatially tight beam. Obtained here far-field energy characteristics ($152^{\circ}$ and 4 kW/m ) 
are in a good agreement with previously reported altimeter and in-situ observations. Though, the 
numerical experiments points to spatial variation. These modulations arise from interference with 
intricate wave field setting in due to reflection from Tasmania and multiple generators found 
throughout Tasman Basin. Among experiments variability arises both due to spatial changes of 
production hot spots along Macquarie Ridge and interaction with mesoscale eddies found near 
Tasmania. Effect of the former process is estimated by a semi-analytical model. Besides a direct 
dependence of baroclinic tide amplitude on conversion magnitude, its spatial distribution can 
modulate beam heading by $\pm 3^{\circ}$ which results in far-field position shifts. (No ending, no 
strong statement so far).
Macquarie Ridge, Tasman Basin where the, where not.\\
%This might lead to large variation of beam's magnitude. In the far-field beam's can also be 
%refractive by eddy field. All of these processes are discussed in terms of the field observations. 

\section{Introduction}
Baroclinic semidiurnal tides originate as a strong barotropic flow along topography forces heaving 
of isopynal surfaces. This process renders scattering of barotropic tidal energy into baroclinic 
motions \citep{hendershott1981long}. The so dispersed energy constitutes a third of global budget 
for lunar semidiurnal constituent \citep{egbert2000significant, munk1997once} and contributes 
significantly to internal wave climate (ref?). At most conversion sites because of highly inclined 
slopes, internal waves of tidal period (internal tides) are radiated as low baroclinic mode. Due to 
their large length scales decay rates are subtle. This makes low mode tide efficient in carrying 
baroclinic energy (kinetic?) over distances comparable to size of ocean basins. While generation 
sites were identified \citep{morozov1995semidiurnal, simmons2004internal, arbic2010concurrent, 
zhao2016global}, some 
were studied in detail \citep{rudnick2003tides, klymak2011breaking, althaus2003internal} and 
analytical models have been developed \citep{garrett2007internal}, little is known on how fast and 
where internal tidal energy is dissipated (deposited). A lot of uncertainty arises because of close 
relation between the waves and the dynamical oceanic medium, so the wave field is subject to 
continuous change.\\
Water column stratification directly impacts internal wave dispersion. In fact, analytical models 
of generation emphasize a ratio between angle of internal wave characteristics to bathymetric slope 
\citep{garrett2007internal} along with a height of topography as primary quantities in setting 
conversion levels \citep{llewellyn2003tidal, petrelis2006tidal}. For tall, steeply inclined 
submarine ridges the energy transfer approaches an upper theoretical limit 
\citep{petrelis2006tidal, st2003generation} making them to be ``oases" of barotropic tide 
scattering 
and internal tide production \citep{morozov1995semidiurnal, egbert2000significant}. Clearly, the 
energy rates can be modulated by changing buoyancy frequency \citep{holloway1999internal}, 
especially when seasonal transformation of water properties happens at same depths as the steepest 
bottom gradients 
\citep{gerkema2004internal}. Nevertheless, in later field studies it was realized that presence of 
external baroclinic tidal signal leads to even larger temporal variability \citep{Kelly2010a, 
zilberman2011incoherent, pickering2015structure}. This might occur as opposite ridge slopes 
affect each other \citep{nash2004internal, zilberman2011incoherent, echeverri2010internal} or due 
to spatially inhomogeneous distribution of production hotspots \citep{osborne2011spatial, 
ponte2013coastal}\footnote{there should a verb here, needs to be restructured}, or as separate 
topographic features mediate each others generation energy levels \citep{xing1998three, 
buijsman2012modeling, buijsman2014three}. This study addresses temporal and spatial variability of 
tide production happening at Macquarie Ridge, south from New Zealand. Quick recourse to a map of 
Tasman Sea (\fignm{C3:fig:geo.map}) suggests that location of major sea bottom features leads to 
complex internal tide regime representative both of Kaena Ridge and Luzon Strait.\\
Macquarie Ridge emits energy forming a spatially confined beam \citep{simmons2004internal, 
zhao2016global}. This is an ubiquitous characteristic of low mode internal tide propagation in the 
deep ocean that is thought to be a result of multiple source interference 
\citep{rainville2010interference}. The Tasman beam carries away most of the 
conversed energy and partly deposits it on Tasmanian continental slope found $\sim 1000~km$ from 
the ridge. To detail contributing (concurrent) physical processes several field experiments 
(TBEAM/TTIDE/Tshelf) were conducted \citep{pinkel2015breaking} along with an investigation of 
satellite altimetry observations \citep{zhao2018satellite}. The latter results were favorably 
compared to averaged in-situ measurements \citep{waterhouse2018observations} corroborating 
(existence of?) northwesterly propagating low mode beam of small decay rate. Nevertheless, the 
observed temporal variability of the beam's heading and amplitude needs an interpretation to 
restrain boundary conditions for a problem of shoaling (scattering and reflection?) internal tide 
on three dimensional topography and consequent energy dissipation \citep{klymak2016reflection}.\\
The non-stationary behavior in propagation of the baroclinic tidal waves results from interaction 
with varying oceanographic conditions \citep[e.g.,][]{mooers1975several}. Depending on involved 
length scales and magnitudes, different regimes can be realized \citep{buhler2014waves}\footnote{I 
didn't read Buhler, 2014 book, just he discusses 
in great detail the topic}. On the first order, when the (wave-flow) scales are largely detached as 
in  
geometric optics limit, the oceanic conditions simply change mode-1 phase speed and cause wave 
front refraction \citep{rainville2006propagation, zaron2014time, kelly2016internal}. The phenomena 
is augmented in presence of considerable mean flows with vertical structure 
\citep{park2006internal, buijsman2017eq}. This can further produce non negligible Doppler 
shifting 
\citep{chavanne2010surface} and shifts of apparent wave frequency in strong vortical flows 
\citep{kunze1985near}. In the higher orders, nonlinear interactions lead to scattering into 
high modes \citep{dunphy2014focusing}, directional spreading \citep{wagner2017asymptotic, 
dunphy2017low} and nonintuitive energy transfers via resonant triad interactions with geostrophic 
turbulence \citep{ward2010scattering}. Still in typical oceanographic setting the first order 
mechanisms are the most widespread \citep{kelly2016internal, zaron2014time}. The latter 
work as well investigated role of generation in the producing time variable far-field. This was  
 hypothesized by \citep{wunsch1975internal} who suggested that ``energetic beams will be moved 
comparatively large distances by small changes in angle and may be missed by isolated instruments" 
\footnote{this is a direct quote, marks are ok?}.\\
In the setting of Tasman Sea both phenomena are plausible reasons to produce the documented 
variation 
in incidence of the low mode tidal beam on Tasman continental slope. To identify cause-and-effect 
relationship numerical experiments are carried out with different conditions of the oceanic medium 
(Section 2). Variable levels of internal tide production are examined in Section 3a and 
resultant beam's characteristics are quantified in its traverse of Tasman Basin (Section 3b). 
These results are brought together to be studied in terms of a semi-analytical generation 
model and action of mesoscale (Section 4a). This helps to provide context for the field 
observations (Section 4b). This follows by conclusions. And in the Appendices mathematical nuances 
are described in greater detail.

\newpage

\section{Numerical experiments and analysis}
\subsection{Numerical experiments}
To study variability of internal tide generation around New Zealand and its propagation numerical 
simulations were performed with Regional Ocean Modeling System \citep{shchepetkin2005regional}. 
The numerical domain covered southern Tasman Sea from subantractic waters of $60^{\circ}$ S 
to subtropics in $35^{\circ}$ S. And the zonal extent stretched from $142^{\circ}$ to $172^{\circ}$ 
E. This ensued correct representation of reach regional oceanographic conditions. The horizontal 
grid spacing was taken to be of $1/32^{\circ}$ corresponding on average to discretization of 3 km 
in zonal direction and 2.5 km in meridional. The nonuniformly separated, vertical 50 $s$-levels 
were placed to smoothly follow subsurface terrain.\\
\begin{figure}
	\centering
	\includegraphics[scale = 0.35]{../figures/fig_1.png}
	\caption{Domain of numerical simulations with geographical locations used in the text}
	\label{C3:fig:geo.map}
\end{figure}
Such discretization of vertical momentum equation tends to induce artificial, horizontal  
along-slope flows \citep{haidvogel1999numerical} due to errors in reproducing of pressure 
gradient force. Especially severe errors are made by steep terrain. The misbehavior is usually 
solved by aritificial smoothing of topography. This procedure additionally increases numerical 
stability, but has an adverse effect on internal tide generation \citep{di2006numerical} since 
primary production sites are collocated with large topographic gradients. To test the numerical 
setup, a sensitivity study was carried out with simulations of 
$1/8^{\circ},~1/16^{\circ},~1/64^{\circ}$ horizontal resolution. The essential for this study 
internal tide behavior manifested at $1/16^{\circ}$ and converged for $1/32^{\circ}$ and 
$1/64^{\circ}$ cases. There no marked differences were observed, except a substantial increase in 
high mode content which is in line with \footnote{previous investigations} 
\citep{di2006numerical}.\\
This work addresses the gravest baroclinic mode dynamics in the deep ocean. Spatial extent of 
waves is large compared to associated vertical displacements. This ensures linear regime of 
propagation without dispersive and nonhydrostatic effects taken place such as fission into 
solitons. A hydrostatic solver used in ROMS seems to be a proper choice for the simulations. Such 
simplification in wave dynamics was assumed in previous studies (\citep{carter2008energetics, 
merrifield2001generation,  merrifield2002model, kerry2013effects}). In more dynamically accurate   
simulations of \citep{kang2012energetics, zhang2011three} the nonhydrostatic effects are found to 
be important only for internal tides in shallow waters, while for main part generation follows 
linear dynamics with vertical accelerations to have a negligible contribution.\\
The horizontal boundary conditions were imposed to be open for depth-averaged, barotropic flows 
following recommendations proposed by \citep{marchesiello2001open}. The baroclinic fields are 
nudged to zero by linear increased lateral viscosity and diffusivity over sponge layers. Through 
the same outer boundaries numerical simulations were forced with barotropic tide. The tidal 
currents and sea level are derived from TPXO atlas, version 7.2 \citep{egbert2002efficient} and 
prescibed as linearly interpolated volume transports. It was used only the largest semidiurnal 
constituent $M_2$. Amplitude ratio between the principal lunar and solar components are 4-to-1 
suggestive of slight open-ocean spring-neap modulation. The diurnal species are weak in the region 
except shoals east of New Zealand \citep{walters2001ocean}.\\
To investigate variations of baroclinic tide dynamics several ocean states were prescribed and 
analyzed separately. In the simplest setting lateral gradients in water properties were absent, 
while buoyancy frequency was set to representative of Tasman Basin. The second set of simulations 
was comprised to investigate interannual and interseasonal variability (Table 1). And the third 
calculation was intended to cover period of TTIDE/TBEAM/Tshelf field programs 
\citep{pinkel2015breaking}, a single experiment once initialized was left to proceed for three 
numerical months.
\begin{table}
	\caption{Carried out numerical experiments}
	\begin{tabular}{ |p{3cm}||p{5cm}|p{5cm}|  }
		\hline
		\multicolumn{3}{|c|}{Numerical experiments used in this study} \\
		\hline
		Experiment abbreviation & Simulation period & Comments (reason?) \\
		\hline
		Uniform & ~ & No mesoscale \\
		2012 &   Jan 1st - Jan 15th, 2012 & Interannual \\
		2013 &   Jan 1st - Jan 15th, 2013 & Interannual \\
		2014 &   Jan 1st - Jan 15th, 2014 & Interannual \\
		2013\_Oct &   Oct 1st - Oct 15th, 2013 & Interseaonal \\
		2015\_Mar &   Mat 1st - Mar 15th, 2015 & Interseaonal \\
		2015\_TTIDE$^{\ast}$ &   Jan 1st - Mar 1st, 2015 & Field period \\
		\hline
		\multicolumn{3}{|l|}{\footnotesize$^{\ast}$ the results are named as respective day of 
		year over which post-analysis was performed, e.g. $d20-25$ }\\
		\hline
	\end{tabular}
	\label{ch2:table_exp}
\end{table}
The simulations with variable conditions were at first initialized by HYCOM hindcasts 
\footnote{(NAVGEM;	downloaded from hycom.org)} for respective start date. Then during integration, 
along with barotropic tidal flow, time-variable, subtidal two dimensional fields 
\footnote{(vertical coordinate and along boundary coordinate)} of horizontal currents, temperature 
and salinity were imposed onto the numerical ocean. The air-sea interaction obtained from 
MERRA-reanalysis \citep{rienecker2011merra} was also given by insolation, air  temperature, EP 
rates and most importantly, wind stresses.

\subsection{Internal tide analysis}
As it is seen in table 1, the simulations were carried out for 15 days or longer. The first 10 days 
were left for spin up of baroclinic tide generation and propagation. Roughly, it takes about 7 
days for the mode-1 signal to traverse Tasman Sea from New Zealanda to Tasmania. After that 
period, three dimensional fields of velocity, temperature and salinity were sampled hourly. These 
were 
later subject to high pass filtering with Butterworth filter of order $6$ with cut off time of $36$ 
hours. This removed subtidal motions and left out signal was further fit in a least 
square sense to the principle semidiurnal harmonic. Then the three dimensional fields 
underwent a separation into barotropic and baroclinic signals \citep{cummins1997simulation, 
kunze2002internal, carter2008energetics}. A depth-averaged current is thought to represent a pure 
barotropic signal and any vertical deviation is attributed to a baroclinic wave,
\begin{equation}
\label{ch2:bt_bc_vel}
\vec{u}_{bt}(x,y) = \frac{1}{H} \int_{-H}^{0} \vec{u}(x,y,z)  dz,~\vec{u}_{bc}(x,y,z) =  
\vec{u}(x,y,z) - \vec{u}_{bt}(x,y)
\end{equation}
To describe distribution of pressure, at first, from a linear equation of state and respective 
\textit{TS}-fields density perturbation from the reference is found. Then the hydrostatic 
approximation is employed and after vertical integration the total pressure field is found. This 
is then subject to baroclinicity condition, so that baroclinic pressure anomaly is taken to be a 
deviation 
from the depth-averaged,
\begin{equation}
\label{ch2:bt_bc_pres}
p(x,y,z) = \int_{-z}^{0} \rho(x,y,z) dz,~p_{bc}(x,y) = p(x,y,z) - \frac{1}{H} \int_{-H(x,y)}^{0} 
\rho(x,y,z) dz
\end{equation}
In the both expressions rigid-lid approximation is used. This is a valid statement unless vertical 
accelerations are smaller than acceleration due to gravity which is true except shallow depths 
\citep{kelly2010}.\\
Each dynamical variable was then decomposed into vertical modes. The structure functions were 
obtained from local Brunt-Vaisala frequency profiles found from time-averaged density fields. 
These were used in Sturm-Liouville problem for the hydrostatic approximation,
\begin{equation}
\frac{d}{dz}((\frac{\omega^2 - f^2}{N^2} ) \frac{d \psi(z)}{dz}) + c^2_n \psi(z) = 0
\end{equation}
where $c_n$ is the mode phase speed in nonrotating ocean. The first 3 vertical modes were 
fit into three-dimensional fields. And only mode-1 was used in the following results.\\
Now energy diagnostics could be obtained. First, depth-averaged mode-1 energy flux is
\begin{equation}
\vec{F} = \frac{1}{2} \frac{1}{H} \cj{\vec{u}} p \int_{-H}^{0} \psi_1(z) \psi_1(z) dz
\end{equation}
At second, rates of conversion from barotropic to baroclinic \citep{simmons2004internal, 
kurapov2003m} were calculated as
\begin{equation}
\label{C3:eq.conv}
C_{bt\to 1} = -\frac{1}{2}(\cj{\vec{u}_{bt}} \cdot \nabla H) p_{1,~bot}
\end{equation}
The fraction $\frac{1}{2}$ in front of the energy characteristics appear because harmonic, complex 
amplitudes are used in the expressions.\\
These calculations had produced a set of dynamical variables of barotropic and baroclinic 
fields in each experiment. The obtained values per experimenter hereafter will be referred as a 
realization. For instance, the longest experiment, 2015\_TTIDE had 10 realizations. To study 
variability of the system, mean values were defined as arithmetic mean,
\begin{equation}
<\bullet> = \frac{1}{N} \sum_i \bullet_i
\end{equation}
where $\bullet$ a field being averaged and $N$ is a number of experiments used. The variation 
between realizations is studied by mean deviation,
\begin{equation}
\Delta \bullet = \frac{1}{N} \sum_i (\bullet_i - <\bullet>)
\end{equation}

\subsection{Discrete Fourier Decomposition by inverse modeling} \footnote{I will move it to 
Appendix and will leave just a paragraph or two}
In addition to the above characteristics the mode-1 internal tide field was subject to directional 
analysis in order to remove interference modulations. Similar methods 
were used previously in internal tide field programs \citep{hendry1977observations, 
lozovatsky2003spatial} \footnote{that were based on array beamforming method and stationarity of 
the field} or 
satellite altimetry \citep{dushaw2002mapping} or in surface wave studies 
\citep{longuet1961observations, munk1963directional, long1986inverse}. Let 
mode-1 pressure in complicated seas to be described by an angular spectrum
\begin{equation}
\label{C1:eq.spectrum}
p(\vec{r}, t) = \int_0^{2\pi}  d\theta_k S(\theta_k) e^{i \vec{k}(\theta_k) \cdot \vec{r} + 
\phi(\theta_k) - i \omega t}
\end{equation}
Here each elementary (monochromatic) sine wave of wavenumber $k$ travels in direction $\theta$ with 
energy $S(\theta)^2 d\theta$ and temporal (spatial) lag of $\phi(\theta)$. The statement can be 
reformulated in terms of Fourier coefficients \citep{munk1963directional} by application of 
Jacobi-Anger expansion,
\begin{equation}
p(r, \theta) = e^{i \vec{k}(\theta) \cdot \vec{r}} = \sum_{m = -\infty}^{m = \infty} i^{m} J_{m}(k 
r) e^{im(\theta - \theta_k)}
\end{equation}
shows that a field at point $(r, \theta)$ produced by plane wave can be expanded in series of 
Bessel functions and circular functions. Then its substitution into \eqref{C1:eq.spectrum} and 
reorganization lead to
\begin{equation}
\label{C1:p.eq}
p(r, \theta) = \sum_{m=-\infty}^{m=\infty} \big[ \int_0^{2\pi}  d\theta_k S(\theta_k) 
e^{i\phi(\theta_k)} e^{-im\theta_k} \big] i^m J_m(kr) e^{im\theta}
\end{equation}
Term in brackets (square brackets) represent convolution integrals defining Fourier coefficients of 
order $m,~A_m - i B_m$. Thence, series \eqref{C1:eq.series} state a model equation to find the 
unknown coefficients from the known, measured pressure field that were sampled at a set of points 
$(r_i, \theta_i)$ and if infinite series is truncated at some order $N$. Real and imaginary parts 
will constitute two separate problems allowing deterministic definition of the spectrum.\\
The same steps are repeated but with current velocities instead. Plane wave  
polarization relations \citep[e.g.,][]{muller2000scattering} are inserted into 
\eqref{C1:eq.spectrum} and the following equations are found,
\begin{align}
\label{C1:uv.eq}
\begin{Bmatrix}
u_i \\ v_i
\end{Bmatrix}
= \frac{1}{2} \sum_{m = -N}^{m = N} J_{m} (kr_i) e^{im(\theta + \pi/2)}
\begin{Bmatrix}
(\omega - f) A_{m + 1} + (\omega + f) A_{m - 1} - i [(\omega - f) B_{m + 1} + (\omega + f) B_{m - 
1}] \\ 
(\omega - f) B_{m + 1} - (\omega + f) B_{m - 1} + i [ (\omega - f) A_{m + 1} - (\omega + f) A_{m - 
1}]
\end{Bmatrix}
\end{align}
The dependence of currents on wave bearing causes splitting of Fourier coefficients and 
asymmetry via Coriolis effect. This results points out that to describe velocity field 
higher circular harmonics have to be used. Physically, velocity field has higher spatial 
wavenumber. But in \eqref{C1:uv.eq} additionally, the asymmetry is observed 
for clockwise and counterclockwise components.\\
The inverse model combines dynamical relations of \eqref{C1:p.eq} and \eqref{C1:uv.eq} into a 
matrix equation
\begin{equation}
y = K x
\end{equation}
Generally, it is unstable to small errors in data and produce physically inconsistent results. This 
can circumvented by seeking a damped least square solution \citep{munk2009ocean} where a 
minimization function is given by
\begin{equation}
\label{C1:Tikh_prob}
J = ||K x - y||^2_2 + \alpha ||x||^2_2
\end{equation}
The unknown regularization parameters $\alpha$ acts as a high-pass filter in a singular value 
decomposition of $K$ \citep{bennett1992inverse}. In field studies this is usually set by a 
signal-to-noise ratio \citep{munk2009ocean}, since the parameter scales noise variance (residue) 
to actual signal's strength. To obtain $\alpha$ in data-driven way a straightforward approach is 
adapted that based on 
trade-off curve method \citep{hansen1993use}. In \eqref{C1:Tikh_prob} amount of allowed error 
is competing with solution's variance. An optimal parameter should balance these factors. This is 
seen as a rapid change in behavior of curve associating residue with model's norm as regularization 
varies. In most cases the curve has a sharp corner connecting aforementioned limits, hence, the 
method's name is a L-curve \citep{hansen1999curve}. And the corner is to occur for an optimal 
regularization parameter.\\
The equations \eqref{C1:p.eq} and \eqref{C1:uv.eq} are sampled at locations in a concentric 
arrays placed at $\lambda,~0.5\lambda,~0.25\lambda$ where $\lambda$ is a local mode-1 wavelength. 
At each location $u,~v,~p$ are used as data and for a region embraced by array Fourier coefficients 
are found. And these then are used in reconstructions.\\
The method used here is different from \citep{zhao2010long} for two main reasons. The model 
equations produce simultaneous fit of all the components, rather than a finite number of a single 
directed plane waves. This can make a difference in regions where diffraction is important such as 
near internal tide generation or scattering regions. And at second, velocity field is utilized 
which provides an additional constrain. Moreover, in synthetic experiments with 
\eqref{C1:Tikh_prob} where instead of $L2$-norm regularization it was used $L1$-norm, the results 
were approaching one of plane wave technique of \citep{zhao2010long}. Additionally, the proposed 
method can be utilized for a single mooring where half-space separation is necessary.

\section{Results}
\subsection{Generation of internal tidal beam}
Surface tide arrives to Southern Tasman Sea from North (\fignm{C3.fig:BT}). Its 
advancement happens in counterclockwise manner with maximum amplitude of sea level located along 
New Zealand's coast. This is a typical Kelvin wave behavior \citep{walters2001ocean}. Also the 
barotropic tide produces strong currents in shallow Bass Strait, but relatively weak anywhere else 
in the basin. The simulated sea surface tidal oscillation closely follows TPXO atlas with gross 
features well captured. Presence of baroclinic field manifests in perturbation of sea level 
magnitude and cotidal lines. In the basin this have a striking wavy character. This corresponds to 
propagation of low mode tidal wave (\fignm{C3.fig:beam}).\\
The baroclinic tidal field represents a complex pattern produced by multiple generation sites 
defined by steep topography. Primary production sites are just south of New Zealand. Here 
barotropic Kelvin wave faces prominent Macquarie Ridge stretched for 2000 km. As barotropic current 
decays away from the coastline conversion lessens as well. Nevertheless, low-mode beams are emitted 
from many locations. Major conversion happens at $49.5^{\circ}$ shedding away the strongest beam. 
This and two nearby beams were identified in altimetric observations \citep{zhao2018satellite}. 
Henceforth, analysis is concentrated on the most energetic, central beam.\\
The central beam is produced by tidal currents impinging on supercritical bathymetry 
(\fignm{C3.fig:gen}). Depth of the highest conversion is between 1000-3000 m and spatially confined 
to two seamounts that are separated with a sill. It has less inclined slopes and plays 
lesser role in production of the tidal beam. On average, this region of Macquarie Ridge converts 
1.6 GW of surface tide. This is half of production of Kaena Ridge, Hawaii 
\citep{carter2008energetics} and much less than Luzon Strait \citep{}. Pattern of conversion 
(\fignm{C3.fig:gen}) also exhibits regions of internal tide destruction produced by complex 
dynamics caused by superposition. In fact, an oppositely located Aucklands Escarpment presents an 
important 
source of 
baroclinic energy. \fignm{C3.fig:beam} clearly illustrates existence of a standing wave in 
Solanders Trough. The total field is characterized 
by a node in horizontal kinetic energy with fluxes revolving in counterclockwise direction 
\fignm{C3.fig:stand_wave} because of Southern Hemisphere. By the method proposed in 
ref-to methods, the standing wave is separated into elemental east-west directed components 
(\fignm{C3.fig:stand_wave}). On leeward side of Macquarie Ridge generation occurs at the same 
seamounts but there is a region of destruction that coincides with incidence of waves emitted by 
the escarpment. There generation has reacher structure both due to more complicated 
topography that is crisscrossed by canyons but also because of Macquarie ridge produced waves. 
Quantifying energy transfer across the trough and comparing with spatially integrated conversion 
rates points out to fact that wave energy is being recirculated by slope's supercritical 
reflection and only partly fed by barotropic field. In overall, such system is similar to Luzon 
strait where resonance conditions exist between two parallel ridges \citep{buijsman2014three}. In 
case of Macquarie Ridge and Campbell Plateau resonance is only partial since the former has a slant 
orientation of $15^{\circ}$. Though at $49.5^{\circ}$ the distance corresponds to 3/4 of mode-1 
wavelength. Such spacing is sensible to phase lags and can either lead to intensification or 
destruction of generation. This is illustrated by comparison of two simulations that presents 
dynamically different regimes of generation.\\
%While \eqref{C3:eq.conv} provides a convenient way to quantify energy transformation, it does 
%not have much room for physical interpretation. Such as in complicated situation when along with 
%local baroclinic tide production a remote signal is present, resultant perturbation of bottom 
%pressure might lead to rather ambiguous result of internal tide destruction. It is said that such 
%regime is to occur when a phase difference between $w_{bt}$ and $p_{1}$ is in range of 
%$(\pi/2,~3\pi/2)$. This is understood as an internal tide performing work against barotropic 
%forcing. Unfortunately, this statement does not directly follow from \eqref{C3:eq.conv}. Hence, to 
%provide cleaner physical picture, let derive expression for conversion rate from the first 
%principles. Body force of \citep{baines1982} is performing work by displacing isopycnal surfaces 
%throughout a water column,
%\begin{equation}
%\label{C3:eq.convd1}
%C(z) = \frac{dW(z)}{dt} = F_{B} w_{bc} = \frac{N^2 (-\vec{u} \cdot \nabla h) z}{i \omega h} 
%\frac{d 
%\xi}{dt} = \frac{w_{bt}}{i \omega h} \frac{z d(-b)}{dt} = -\frac{1}{h} w_{bt} zb
%\end{equation}
%where isopycnal displacements $\xi$ were changed to buoyancy, $b = -N^2 \xi$ and temporal 
%variation 
%was assumed to be harmonic, $\sim e^{i \omega t}$. On a final step, integration by parts can be 
%employed as,
%\begin{equation}
%\label{C3:eq.convd2}
%\int_{-h}^{0} z b dz = \int_{-h}^{0} z d \big( \int^0_{z} b dz \big) = \big( z \int^0_{z} b dz 
%\big)\big|_{-h}^0 - \int_{-h}^{0} dz \int_{z}^{0} b dz^{\prime} = h (\int_{-h}^{0}b dz - 
%\frac{1}{h} \int_{-h}^{0} dz \int_{z}^{0} b dz^{\prime})
%\end{equation}
%The last expression is a bottom pressure perturbation. Noteworthy, a baroclinicity condition was 
%not employed. Combining \eqref{C3:eq.convd1} and \eqref{C3:eq.convd2}, familiar result 
%for conversion rate is obtained. Note that a similar approach to \eqref{C3:eq.convd1} was used 
%by \citep{nash2006structure} to estimate an upper limit on emitted energy.\\
Conversion rate shows how much work was done by baroptropic tide to displace isopycnal interfaces. 
This is understood as work against buoyant forces. But it can happen that barotropic forcing will 
be oppositely directed if somewhere in the water column other forces are present. For case of 
'2014' 
simulation (\fignm{C3.fig:gen_2d}, a-b) during ebb tide on tideward side there is net energy 
conversion to baroclinic field even that along bottom an internal wave ray is developing by upward 
displaced interfaces as a result of previous tidal phase. Hence, at these location a newly 
generated internal wave does work against downward barotropic flow. And this produces negative 
conversion (\fignm{C3.fig:gen_2d}, e) at some moment. As tide turns conversion changes sign 
again. Overall, period averaged transfer is positive, i.e. surface tide losses energy. In 
the other experiment presented (d10-15, \fignm{C3.fig:gen_2d}, c-d), there is an intensification 
due 
to advancement of a mode-1 wave. In actuality, its propagation from Solanders Trough 
($165^{\circ}$) 
and over the sill is the major difference between two simulations. So in '2015'-setting similar 
along slope advancement of an internal wave ray is observed, but now due to shoaling mode-1 
conversion is positive throughout tidal cycle.\\
The contrary situation is found on leeward side ($164.5^{\circ}$) where surface tide current has the
opposite direction. For '2015' it appears that propagating mode-1 is losing energy since it does 
work against barotropical forcing by dipping isopycnals. Though in transition some energy is lost 
from surface tide. In '2014' there is a reflection of an internal wave ray as phase is advancing 
onto the sill. This coincides with upward barotropic flow that in total leads to intensification. 
In total, period averaged conversions have different signs on leeward side. Comparison of mode-1 
wave 
position in the trough at ebb tide (\fignm{C3.fig:gen_2d}, a,c) suggests difference in timing of 
the 
remote wave arrival or its advancement. This is explored on (\fignm{C3.fig:gen_2d}, g). The total 
signal shows a region of low phase change eastward of $165^{\circ}$ that corresponds to 
concentration of kinetic energy (\fignm{C.3:stand_wave}, a). Decomposition of the signal presents 
it roughly as a sum of the ridge generated waves (eastward propagation) and the escarpment 
originated (westward) waves. The actual relation will depend on relative magnitudes 
\citep{martini2007diagnosing}. 
But westward of $165^{\circ}$ there is almost a free propagation in '2015' of the ``escarpment" 
waves as the total signal closely corresponds to them. But in '2014' due to sills generation and  
strong reflection, phase difference with local baroptropic tide falls below 
$90^{\circ}$, so that there is an intensification of generation.\\
The described situation provides a several approaches to narrate variability in conversion rates. 
At first, it is clear that amount of remote energy crossing Macquarie Ridge through the sill will 
shape the overall conversion. Hence, energy incident from leeward side can provide such 
estimate. It is quantified from the numerical simulations by line-integration of energy fluxes 
through leeward side (\fignm{C3.fig:gen}). Additionally, this amount is modulated by reflection of 
the 
sill. Here it is thought as a knife-edge barrier for which reflectivity was analytically found by  
\citep{larsen1969internal} with similar investigations of \citep{klymak2013parameterizing}.  
Sill's depth is obtained as a depth jump from the trough to the sill that were WKB-scaled. The 
resultant calculation is given by (\fignm{C3.fig:gen_regr}, a) where total conversion of tideward 
side 
of Macquarie Ridge is plotted against amount of transmitted mode-1 energy. At second, efficiency 
of generation will depend on phase difference between the remote waves and local forcing. This is 
estimated from mode-1 eigenspeed for Solanders Trough. Taken distance separating Macquarie Ridge 
and Aucklands Escarpment to be about $115~km$ (average mode-1 wavelength $155~km$), the time to 
cross the trough can be found (\fignm{C3.fig:gen_regr}, b). And additional factor is environmental 
changes that are associated with overall efficiency of generation by the ridge. Again by applying 
WKB-scaling variation of ridge's depth relative to the surrounding deep ocean is found and then 
scaled to converted energy by application of theory by \citep{st2003generation}. The 
mean barotropic current of $0.03~m/s$ generation was considered to produce 
(\fignm{C3.fig:gen_regr}, 
c).\\
The linear regression for the first parameters has correlation coefficient $R^2$ slightly higher 
than 0.5, and for linear multiple variable regression it increases to $0.7$. The least dependence 
is found for stratification variability on the tideward side. Though if all three parameters 
combined $R^2$ becomes $80\%$. Hence, most variability is associated with leeward dynamics and not 
local stratification. Additionally, all three environmental parameters do not always change in 
similar fashion suggestive for different mesoscale dynamics occurring in the deep sea, over 
Macquarie Ridge and in Solanders Trough. This is not surprising since the region is affected by 
frontal zone and reach in subtidal dynamics \citep{smith2013interaction}. Further, inclusion of 
transmission coefficient increases correlation, but most variability comes from amount of energy 
traveling from Aucklands Escarpment 
(\fignm{C3.fig:stand_wave}, c). This is much harder problem to estimate since generation in that 
region has much more spotty character (\fignm{C3.fig:gen}) because of complex topography and large 
influence of eastward traveling waves which can also act either to intensify local generation or 
destroy baroclinic tides.
%So in total, generation at Macquarie Ridge is occurring in 
%complex dynamical environment that largely modulated by ocean conditions. 


%Generation of baroclinic tide primarily happnes south of New Zealand. 
%in circular fashion 
%Southern Tasman Sea represents a case of complicated of 
%internal tidal field. Even baroptropic 
%currents are 
%small and topography is not steep, there are a lot of regions surrounding the Southern Tasman sea 
%to produce internal tides (Figure 1). To name a few Southern Tasman Rise, Lord Howe Rise, Bass 
%Strait. Though the primary production site is found at Macquarie Ridge that stretches poleward 
%from 
%New Zealand for almost 2000 km. There several internal tidal beams are emitted along different 
%sections. These were previously identified in satelite altimetry \citep{zhao2016global}. The major 
%beam is produced where most of energy transfer from baroptropic tide to baroclinic field happens 
%at 
%$45^\circ$S.\\
%The steep topography and strong semidiurnal currents produce convergence of barotropic flow. This 
%primarily occurs at two seamounts and along 3000 m isobath. Here the topography is super critical 
%internal waves to be radiated in the low mode signal. The other generation is located by sill 
%connecting two seamounts. This pattern and its spatial distribution is subject to variability. As 
%it is observed there could occur a shift in position.\\
%As it is seen on Figure 1 this region has a complex internal wave field. The changes in generation 
%attributed to interaction of remote tidal waves and locally generated. This is illustrated by two 
%opposite experiments, 2014 and 2015. It is quite obvious to see marked difference in Solander 
%trough. On tidal flood stage barotropic tide is climbing up flanks of Macquarie Ridge causing 
%development of vertical velocity. At the same time, in 2015 there is a propagation of mode-1 
%internal wave which then slams generation. On the opposite, generation at 2014 shows beam pattern, 
%so there is a direct generation of internal wave. This is then emphasized by period-averaged 
%conversion. It shows different signs of conversion.\\
%To study these difference we perform directional decomposition (Figure). The total field is 
%comprised of two waves oppositely directed: one generated on lee side of Macquarie Ridge and the 
%other at Aucklands Escarpment they result in a superposed field. Rather than employing mechanism 
%of 
%resonance it is more apparent that the total field and consequent energy transfer happens 
%depending 
%on amplitudes of the mentioned waves and due to different slanting angles. In the simplest case of 
%equal ratio there is no energy transport and portion of it is directed towards.\\

\newpage
\subsection{Characteristics of Tasman tidal beam}
Widespread energy conversion at Macquarie Ridge produce a clearly defined internal tidal beam 
(\fignm{C3.fig:beam}). To delineate spatial variability of energy characteristics, their cross-beam 
averages are introduced (\fignm{C3.fig:beam}); and to investigate respective temporal 
variations, ensemble-means and standard deviations are reported. It is worth noting that while 
ensemble-mean allows a straightforward interpretation of mode-1 dynamics, it will incorporate a
portion of variable (non-stationary) signal since any energy characteristic is a nonlinear quantity 
\citep{zaron2014time}. On contrary, for instance, a flux found from ensemble-mean pressure and 
currents will exclusively provide a stationary part. Yet if a mode-decomposed signal is considered, 
finding the means will entail averaging of vertical basis functions. As a consequence, their 
orthogonality will not be preserved leading to ambiguity in dynamics. Henceforth, an ordinary mean 
over realizations of beam's energetics is considered. Performed comparison (not shown) between an 
ensemble-mean of flux and a flux of ensemble-mean did not reveal significant differences in spatial 
structure, though magnitude of latter was roughly half smaller.\\
The spatial-temporal changes are provided by \fignm{C3.fig:beam_prms}. Dashed lines mark 
second-order polynomial fits \citep{zhao2018satellite} used to remove along-beam oscillations.
On average, the central beam leaves Macquarie Ridge carrying away 1.6 GW in mainly western 
direction (\fignm{C3.fig:beam_prms}, (a, b)). In the open ocean, over $\sim100~km$ there is a rapid 
directional turn, but as the beam progresses, rate of change decreases. The latter regime is an 
example of refraction \citep{cummins2001north, rainville2006propagation, zhao2018satellite} due to 
a meridional change of Coriolis parameter and consequently, a phase speed. Same phenomena is also 
seen in ratio of HKE to APE (\fignm{C3.fig:beam_prms}, (c)). By Tasmania the beam is 
oriented in northwestern direction ($150^\circ$ from East) and has more available potential 
energy (APE) at the expense of horizontal kinetic form (HKE). Additional feature is an energy decay 
(\fignm{C3.fig:beam_prms}, (b)) that is caused by geometrical effects and by dynamical processes 
such as friction and high mode scattering. Amount of deposited energy was determined by flux 
divergence and on average comprised $35~\%$ so that the beam brings $1~GW$ of baroclinic energy to 
Tasmania.\\
According to the numerical experiments there is a high degree of spatial variability 
(\fignm{C3.fig:beam_prms}). Flux vectors sinuously deviate from the fit by $\pm 30^\circ$. The 
alterations align with growth or decline in the flux magnitude. While the energy partitioning 
reversely follows this arrangement. These aspects of propagation are indicative of interference 
\citep{martini2007diagnosing, zhao2010long} between the internal tidal beam and remote waves. 
\fignm{C3.fig:beam_dcmp} attest presence of multi-directional wave field. Its components were 
extracted by the same method of directional spectra (Appendix A) that was then integrated over 4 
circular quadrants. Then for each an ensemble-mean was obtained. The northwesterly traveling waves 
highlight a plane-wave propagation of the internal tidal beam. This component will be 
referred as a planar beam. Its total energy is partitioned closely following a theoretical 
prediction. However, contrary to expected equatorward increase of potential energy, there is a 
decay that pertains to frictional losses. \footnote{Minor oscillations are artifacts of analysis 
owing to finite size of fitting windows. } Southwesterly waves (\fignm{C3.fig:beam_dcmp}, (b)) 
originate from Lord Howe Rise and Gilbert Seamount. Both are oriented against the surface tide 
current (\fignm{C3:fig:geo.map}) and hence, efficiently scatter the barotropic energy (e.g., 
\fignm{C3.fig:beam}). Macquarie Ridge also emits waves traveling in this quadrant. As these signals 
interfere with the planar beam,  the aforementioned rapid directional turn close to the ridge can 
now be easily understood. Further, APE becomes only slightly perturbed, but the ratio clearly 
demonstrates $\sim 200~km$ modulations. Overall pattern is further complicated with an addition of 
southeasterly signals. This introduces another spatial frequency of $\sim 150~km$ that is affirmed 
by spatial variation in the potential energy. This irregularity is limited to region $\sim500~km$ 
from Cascade Seamount. Because this topographic feature reflects portion of the internal tidal beam 
and cause existence of SW waves. Note their tenfold weakness to the planar beam. On the 
last panel additional source of baroclinic tides is discovered at South Tasman Rise. Obviously, 
presence of the multiple generation sites leads to existence of ``internal tidal swell" 
\citep{hendry1977observations} which is not strong, but yet obscures the beam propagation.\\
The interference will inevitably produce signature in the temporal variability 
(\fignm{C3.fig:beam_prms}, purple lines and thin gray lines). Large changes in energy 
partitioning is collocated with locations of nodes-antinodes. The flux heading has similar 
pattern, but degree of variability rises as the tidal beam crosses Tasman Basin. Oppositely, the 
integrated flux alters less between the simulations and any differences are uniformly distributed 
through the course of propagation. These observations are detailed by \fignm{C3.fig:beam_dcmp_cm} 
where previous reconstructions are given with addition of temporal variability informed 
by along-beam distribution of standard deviations and energy flux variance ellipses. The planar 
beam has mainly magnitude variation that is associated with different conversion levels 
at Macquarie Ridge. With distance deviations become apparent in energy flux orientation, so that 
variability ellipses are becoming slanted to the mean vectors. A striking feature is in uniformity 
of the ellipse distribution and mean vector orientation. This behavior alludes to changes in 
generation as a cause. Further, inclusion of the swell produces nodes-antinodes so that now energy 
density variability concentrates near these locations. Contrary, regardless of spatial modulation 
in the mean flux vector heading, its variability preserves uniformity. But this characteristic 
is smeared by the reflected, southeasterly waves. Superposition leads to strong modulations 
in the deviations. Notably, the magnitude variability is not affected on average. Overall, 
the total field demonstrates high degree of spatial-temporal alterations. Nevertheless, temporal 
changes accord with the planar beam behavior and all further complexity arises due to multiple-wave 
interference.

\section{Discussion}
The spatial-temporal characteristics present a complication for direct interpretation of 
diagnosed energy quantities in describing the tidal beam. Its properties are obscured due to 
interference and hence, needs additional inferences. This is related to variability. It is unclear 
what can produce beam's energy levels and its orientation/position. Two reasons could be named as 
accumulating interaction with mesoscale field and generation producing.\\
Now comparison of the decomposed planar beam can be made 
with factual observations made by \citep{waterhouse2018observations, zhao2018satellite}. \\

\begin{table}
	\caption{Internal tidal energy flux properties in the far field}
	\begin{tabular}{ |p{7cm}||p{4cm}|p{4cm}|  }
		\hline
		Source & Flux magnitude [$kW/m$] & Heading [$^\circ$] \\
		\hline
		This study - all experiments & ~ & No mesoscale \\
		This study - only field period & ~ & No mesoscale \\
		Altimetry observations \citep{zhao2018satellite} & $3.9 \pm 2.2$ & $141 \pm 2$ \\
		Field observations \citep{waterhouse2018observations} &   $3.4 \pm 1.4$ & $149 \pm 3$ \\
		\hline
	\end{tabular}
	\label{ch2:table_exp}
\end{table}
\section{Conclusions}

%\newpage
%\iftoggle{only_Chapter} {
%	\appendix
%}
%
%\nottoggle{only_Chapter} {
%	\addcontentsline{toc}{section}{Appendices}
%}
%
%\section*{Appendices}

%\renewcommand{\thesubsection}{\Alph{subsection}}
%\setcounter{subsection}{0}
%\subsection{Analytical model for knife edge}
%Let consider a simplified problem of the internal tide generation at a three dimensional ridge. 
%That is generating topography has extent in along x-axis, $a$, but infinitely small width. We will 
%not pursue full solution of the problem and will not seek actual amplitudes of baroclinic modes, 
%but rather concentrate on defining spatial pattern of the generated waves (might move up to 
%opening). Under such problem statement the actual height is of no importance. And hence, in frame 
%reference traveling with barotropic current, the topography becomes a piston-alike wavemaker that 
%sends out the lowest mode internal tide. Its behavior in flat bottom ocean is well represented by 
%Laplace tidal equations (cf \cite{kelly2012cascade}):
%\begin{align}
%\vec{u}_t + 2 \Omega \vec{k} \times \vec{u} = - \frac{1}{\rho_0} \nabla \cdot p\\
%\nabla \cdot \vec{u} = -(\frac{1}{N^2 - \omega^2}p_z)_{tz}
%\end{align}
%with boundary conditions on the ridge,
%\begin{equation}
%\vec{u}\cdot \vec{n}|_{ridge} = \vec{u}_{bt} \cdot \vec{n} |_{ridge}
%\end{equation}
%And boundary condition in the infinity implying outgoing waves. Note that barotropic current, in 
%general, is given as a current ellipse written in complex form as $\vec{u}_{bt} = u_{bt} + i 
%v_{bt}$. This means that the boundary condition does not have a simple harmonic form. But the 
%barotropic current can be decomposed to clockwise (CW) and counterclockwise components,
%\begin{equation}
%\vec{u}_{bt} = W_{CW} e^{-(i \omega t - \phi_{CW})} + W_{CCW} e^{(i \omega t + \phi_{CCW})}
%\end{equation}
%and the generation of internal tide can be solved separately for two oppositely rotating currents. 
%In further discussion the only one component (CCW) is taken care of since a solution will have 
%similar form with differences arising in sign in front of tidal frequency.\\
%Considering rigid lid approximation and impermeable bottom equations for eigen modes is solved 
%such that wavelength of corresponding mode is introduced. Than the dynamical equations can be 
%reformulated as Helmholtz equation with harmonic temporal dependence implied and appropriate 
%polarization relations,
%\begin{align}
%\nabla^2 p + k_n^2 p = 0\\
%u = \frac{-i \omega p_x + f p_y}{\omega^2 - f^2},~v = \frac{-i \omega p_y - f p_x}{\omega^2 - f^2}
%\end{align}
%Since the problem is now formulated in terms of pressure only, boundary condition takes the 
%following form (\cite{greenspan1968theory}),
%\begin{equation}
%\Big[ \frac{-i \omega p_y - f p_x}{\omega^2 - f^2} \Big]_{ridge} = \vec{u}_{BT}|_{ridge}
%\end{equation}
%The above equations (5-8) state generation of internal tides by vibrating strip of width $a$. Such 
%problem was solved for radiation of acoustic waves by \cite{morse1946methods}. Since in Cartesian 
%coordinates the strip boundary condition does not allow separation of variables, one can employ 
%elliptic coordinate system,
%\begin{equation*}
%x = \frac{a}{2} \cosh \mu \cos \theta,~y = \frac{a}{2} \sinh \mu \sin \theta
%\end{equation*}
%where edges of the strip will be focii of ellipse, boundary condition will take simpler form,
%\begin{align}
%\Big[ \frac{-i \omega p_{\mu} + f p_{\theta}}{\omega^2 - f^2} \Big]_{\mu = 0} = \frac{a}{2} \sin 
%\theta (\vec{u}_{BT}|_{\mu = 0})
%\end{align}
%The factor $\frac{a}{2} \sin \theta$ arises via conversion from Cartesian to elliptical 
%derivatives. Note change of sign for $p_{theta}$ because opposite growth of $\theta$ argument to 
%$x$ argument. Laplace operator in the elliptical coordinates has eigensolutions (''sloshing 
%modes") 
%in form of Mathieu functions (\cite{stratton2007electromagnetic}) that solution for the emitted 
%waves will have form of,
%\begin{equation}
%p \sim \sum_{j} [Se_j, So_j](h, \theta) [Je_j, Jo_j, Ye_j, Yo_j](h, \mu)
%\end{equation}
%with $Se_j,~So_j$ - angular Mathieu functions of order $j$ corresponding to $\cos$ and $\sin$, 
%$Je_j,~Jo_j$ and $Ye_j,~Yo_j$ - radial Mathieu functions of the first and second kind of order $j$ 
%corresponding to Bessel functions. The physical parameter that sets Mathieu functions behavior is 
%$h = {a k_n}{4} $ so that in large distance limit, $h \mu \gg 1,~rk_n \gg a$, they will represent 
%circular harmonics.\\
%Now RHS of boundary condition (9) can be expressed in series of Mathieu functions,
%\begin{equation}
%\sin \theta = \sum_{j = 0}^{\infty} C_{2j + 1} So_{2j + 1} (\theta)
%\end{equation}
%with coefficients $C_{2j + 1}$ defined by normalization constants $N_{2j + 1} = \int_{0}^{2\pi} 
%So_{2j + 1}^2(\theta) d \theta$ and the first Fourier coefficient of $So_{2j + 1}$. The above 
%series for $\sin$ and form of the boundary condition suggest solution in the form,
%\begin{equation}
%p(\mu, \theta) = \sum_{j = 0}^{\infty} \big( A_{2j+1} So_{2j + 1}(\theta) + B_{2j+1} Se_{2j + 
%1}(\theta) \big) Ho^{1}_{2j + 1}(\mu)
%\end{equation}
%Here $Ho^{1}_{2j + 1}(\mu) = Jo_{2j + 1} + i Yo_{2j + 1}$ is a Hankel-Mathieu function. For CCW 
%component to ensure condition for outgoing radiation, the sign should be different, so $Ho^{2}_{2j 
%+ 1}(\mu) = Jo_{2j + 1} - i Yo_{2j + 1}$.\\
%Now substituting (11, 12) into (9) the unknown coefficients $A,~B$ are obtained,
%\begin{align*}
%(-i\omega (\sum A So + B Se) Ho^{\prime} + f (\sum A So^{\prime} + B Se^{\prime}) Ho) = \sum 
%(\omega^2 - f^2) C So
%\end{align*}
%At first, multiplying above equation by $So_{2m + 1}$ and taking integral from 0 to $2 \pi$, so 
%that $\int_{0}^{2\pi} So_{2m + 1} Se_{2j + 1} d \theta = \int_0^{2 \pi} So_{2j + 1}^{\prime} 
%So_{2m 
%+ 1} d \theta = 0,~\int_0^{2 \pi} So_{2m + 1} So_{2m+1} d \theta = No_{2m + 1},~\int_0^{2 \pi} 
%So_{2m + 1} Se_{2j + 1}^{\prime} d \theta = -{N^{\prime}}^{2m+1}_{2j + 1}$. In the last statement 
%orthogonality between $So$ and $Se^{\prime}$ is not satisfied. Equation on each order is obtained
%\begin{align}
%(-i \omega A_{2m + 1} No_{2m + 1}^o Ho_{2m + 1}^{\prime} + f \sum_{j} B_{2j + 1} 
%{Neo^{\prime}}^{2m + 1}_{2j + 1} Ho_{2j + 1}) = C No_{2m + 1}
%\end{align}
%And at second, carrying out the same procedure but with $Se_{2j + 1}$,
%\begin{align}
%(-i \omega B_{2m + 1} Ne_{2m + 1} Ho_{2m + 1}^{\prime} + f \sum_j A_{2j + 1} {Noe^{\prime}}^{2m + 
%1}_{2j + 1} Ho_{2j + 1}) = 0
%\end{align}
%Note that $\int_0^{2\pi} Se^{\prime}_{2j + 1} So_{2m + 1} d \theta = -\int_0^{2\pi} 
%So^{\prime}_{2j + 1} Se_{2m + 1} d \theta$, i.e. ${Noe^{\prime}}^{2m + 1}_{2j + 1} = 
%-({Neo^{\prime}}^{2m + 1}_{2j + 1})^T$.\\
%Equations (13) and (14) form a linear system to find coefficients for different component of the 
%total field. These equations are solved numerically with $j_{max} = 5$ due to rapid convergence of 
%the involved series.
%
%\subsection{Inverse model}
%The model closely follows ideas used in ref-to-Luc, 2010 and -Jody-2016. The internal tide 
%generating ridge is given by point sources each emitting following
%\begin{equation} \label{invm_eq:1}
%p = p_{0} \frac{2}{\pi k d} \cdot e^{i  k  d}
%\end{equation}
%where $k$ - wavenumber associated with eigen mode-1, i.e. $k = \sqrt{\omega ^ 2 - f ^ 2}{c_{eigen} 
%^ 2}$, $d$ - distance between a point source and an observation point. By observation points here 
%and after is meant points in which observations are inverted. The given solution is a solution of 
%pressure distrubance propagation for two dimensional wave equation (p. 22, Frisk) and describes 
%outgoing cylindrical wave. This is a far field approximation ($kd \ll 1$), in the near source zone 
%the solution is substituded by Hankel functions. Here representation is simplified and observation 
%points on the distance less than wavelength are omitted. Though introduction of Hankel function 
%into the inverse model does not involve any additional complexity. By pressure here is thought 
%mode-1 pressure amplitude that can be connected to sea level disturbance or isopycnal 
%displacements.\\
%To describe energy fluxes in the observational points polarization relations for cylindrical 
%Poincare wave are invoked,
%\begin{align} \label{invm_eq:2}
%u = \frac{p_{0}}{\rho_{const}} * \frac{-i \omega \cos(\theta) + f \sin(\theta)}{\omega ^ 2 - f ^ 
%2} \cdot p_{\vec{d}}\\
%v = \frac{p_{0}}{\rho_{const}} * \frac{-i \omega \sin(\theta) - f \cos(\theta)}{\omega ^ 2 - f ^ 
%2} \cdot p_{\vec{d}}
%\end{align}
%where $p_{\vec{d}}$ is a derivative along radius-vector $\vec{d}$,
%\begin{equation}
%p_{\vec{d}} = (i \cdot k - \frac{1}{2 d}) p
%\end{equation}
%In further description of the inverse model it is used following notation, indices $i,~k$ define 
%$i,~k$-th point sources, while $j$ - $j$-th observation point.\\
%The tidally and depth averaged energy fluxes will be given as an interference of pointwise fields 
%from all sources,
%\begin{align}
%F_{j}^x = \frac{1}{2} \sum_k u_{kj}^{\star} \sum_i p_{ij} \int_H^0 \psi_1(z)^2 dz\\
%F_{j}^y = \frac{1}{2} \sum_k v_{kj}^{\star} \sum_i p_{ij} \int_H^0 \psi_1(z)^2 dz
%\end{align}
%Note different indexes for u/v and p meaning that cross multiplication is involved which leads to 
%complex interference pattern. In energy flux formulation normalization coefficient associated with 
%eigenmode structure function are introduced by corresponding mode-1 structure function, 
%$\psi_1(z)$. Coefficient $1/2$ is used for convenience to convert actual time averaging involved 
%to 
%multiplication of complex numbers. In further description the constant coefficients are omitted 
%due 
%to their irreleveance. The previous relations can be expressed in matrix form (it is not fully 
%correct for fluxes, multiplication is done term by term per point),
%\begin{align}
%p_j = B^p_{ji}{p_i},~u_j = B^u_{ji}{p_i},~v_j = B^v_{ji}{p_i}\\
%F^x_j = (B^u_{jk}{p_k})^{\star} B^p_{ji}{p_i},~F^y_j = (B^v_{jk}{p_k})^{\star} B^p_{ji}{p_i} 
%\label{invm_eq:4}
%\end{align}
%where tensor notation is used, i.e. summation is done over same indices. Matrices 
%$B^p_{ij},~B^u_{ij},~B^v_{ij}$ are short notation for generaion model and polarization relations, 
%for example,
%\begin{equation}
%B^p_{ji} = p_i \frac{2}{\pi k d_j} \cdot e^{i  k  d_j}
%\end{equation}
%These can be thought as disretization of operators transforming distribution of sources into 
%interference pattern in pressure and velocity fields.\\
%Apparently, the energy flux relations are non-linear. To deal with this it is proposed an 
%iterative technique. Let at $m$-th iteration there is a known distribution of wave amplitude at 
%sources, $p^m_i$, the total energy flux field can be reconstruced by (\ref{invm_eq:4}). Than it is 
%desired to find a small adjustment $\delta p^m_i$ (``nudge factor") such that residual between 
%observed field and analytical description will be decreased. One can write,
%\begin{align}
%F_{j}^x = (B^u_{jk}(p^m_k + \delta p^m_k))^{\star} B^p_{ji}(p^m_i + \delta p^m_i) = \nonumber\\
%(B^u_{jk} p^m_k)^{\star} B^p_{ji}{p^m_i} + (B^u_{jk} \delta p^m_k)^{\star} B^p_{ji} p^m_i + 
%(B^u_{jk} p^m_k)^{\star} B^p_{ji} \delta p^m_i + (B^u_{jk} \delta p^m_k)^{\star} B^p_{ji} \delta 
%p^m_i\nonumber\\
%F_{j}^x - (B^u_{jk} p^m_k)^{\star} B^p_{ji}{p^m_i} = (B^u_{jk} \delta p^m_k)^{\star} B^p_{ji} 
%p^m_i + (B^u_{jk} p^m_k)^{\star} B^p_{ji} \delta p^m_i + (B^u_{jk} \delta p^m_k)^{\star} B^p_{ji} 
%\delta p^m_i \label{invm_eq:3}
%\end{align}
%The left hand side of (\ref{invm_eq:3}) represents the residual, the right hand side sets a 
%controlling equation to obtain adjustment neceassary to decrease the residual. The last term of 
%RHS 
%shows a non-linear nature of the problem. This is omitted since the purpose of conseqeunt 
%iterative 
%technique is to find the final source distribution such that the model equations (\ref{invm_eq:4}) 
%are satisfied in least square sense. Than the ``nudge-factor" can be found as inverse of 
%\begin{equation}
%F_{j}^x - (B^u_{jk} p^m_k)^{\star} B^p_{ji}{p^m_i} = R_j^x = \Big[ (B^u_{jk} )^{\star} B^p_{ji} 
%p^m_i + (B^u_{jk} p^m_k)^{\star} B^p_{ji} \Big] \delta p^m_i \label{invm_eq:5}
%\end{equation}
%(these equations are not in matrix form, but obsevation point by observation point).
%Hence, the aim of inverse model is to decrease error in representation of energy fluxes. The 
%equation (\ref{invm_eq:5}) can be solved separately for zonal and meridional fluxes and also 
%simultaneously for both directions. That is at each iteration step the nudge-factor is found first 
%for zonal, than for meridional direction and finally, for both simultaneously. At the end pressure 
%distribution is changed by average from all three substeps.\\
%Note the inverse model equation (\ref{invm_eq:5}) is supported by additional condition stating 
%that 
%amplitude is nonnegative, $p_i^m + \delta p_i^m \geq 0$. All of this numerically is solved by 
%linear programming routine \text{lsei} (least square with inequality) provided by LINPACK 
%package.\\
%Here it will be presented a test convergence and number tests on robustness on proposed iterative 
%inverse model. The initial flux field is given by Fig. \ref{invm_fig:1} where by crosses are shown 
%observational points. This define prescribed $F_{j}^x$ or $F_{j}^y$. Note that the prescribed 
%field 
%aims to describe midbasin energy flux field with Tasman shelf ommitted due to presence of 
%reflection and complex bathymetry. The point sources distribution are given by green dots and at 
%the first iteration step are set to $p^0_i = 100 Pa$. The distribution of points sources is 
%representative to distribution of steep bathymetry which is belived to be an internal tide 
%generator. In the inverse model there are only two parameters that describe characteristic of the 
%internal tide, wavenumber and normalization coefficients used in energy flux. Both are found from 
%solving eigenvalue for randomly picked stratification profile. This result in wavelength of 
%$180~km$ which is a representative value for Tasman Sea conditions. In the same way eigenfunctions 
%are obtained and normalization coefficients are found.\\
%Hence, the inverse model does not account for
%\begin{enumerate}
%\item Bathymetry variation
%\item Stratification variability
%\item Variation of barotropic tide along ridges
%\end{enumerate}
%The first two points are thought to have minor effect on internal tidal beam structure. While the 
%third is omitted to preserve simplicity of generation model. Additional tests were done with 
%variation in barotropic tide phase along ridges, but they did not bring any substantial changes in 
%foregoing results.\\
%To show convergence of the inverse model it is given change in pressure amplitude with each 
%iteration. Here convergence is defined by
%\begin{equation*}
%Conv = \sum_i \frac{(p_i^m - p_i^{m-1})^2}{(0.5 \cdot (p_i^m + p_i^{m-1}))^2}
%\end{equation*}
%The iterative solver is stopped when convergence is reaching tolerance. Here it is set to 0.01. 
%From Figure (2a) it is seen that by 17th iteration there is no appreciable change in the inverse 
%solution. This means that influence of non-linear terms in (\ref{invm_eq:3}) became negligible and 
%the distribution of amplitude along the source region is the best in least square sense. The error 
%of such description is given on subsequent panels of Fig. 2, where root-mean-square-error for 
%different energy flux parameters is defined for example zonal component as
%\begin{equation}
%E_{x} = \sqrt{\frac{\sum_{obs} (F_i^x - \hat{F}_i^x)^2}{N_{obs}}}
%\end{equation}
%As it is seen the error is approaching stability for all used components much faster than 
%convergence in amplitude. Note that the error is larger in zonal fluxes. The inverse solution can 
%not predict far field behavior which is believed due to interaction with East Tasman Plateau. The 
%obtained solution is given by Fig. 3. Here it is found that the inverse solution can not well 
%represent the beam close to East Tasman Plateau. The following reasons can be named: interaction 
%with topography and inadequacy of cylindrical wave model in the far-far field. It is believed that 
%the second reason is the main. In general, the inverse solution picks up the central beam pretty 
%well, outlines its boundary and the major region is satisfying manner. As well note that the 
%northern and southern beams are also found in the solution.

\newpage
\section{Figures}

\begin{figure}
	\centering
	\includegraphics[scale = 
	0.5]{/home/dmitry/Work/Research/thesis/FINALE/P3_ITS_GENERATION/figures/fig_1_BT_tide.png}
	\caption{Comparison of $M_2$ sea level oscillations simulated by ROMS (left panel) with 
	TPXO-model (right panel).}
	\label{C3.fig:BT}
\end{figure}

\begin{figure}
	\centering
	\includegraphics[scale = 
	0.5]{/home/dmitry/Work/Research/thesis/FINALE/P3_ITS_GENERATION/figures/fig_2_uni_flux.png}
	\caption{Beam of Tasman Sea with major internal tide production sites identified by superposed 
	heat map.}
	\label{C3.fig:beam}
\end{figure}

\begin{figure}
	\centering
	\includegraphics[scale = 
	0.5]{/home/dmitry/Work/Research/thesis/FINALE/P3_ITS_GENERATION/figures/fig_3_gen.png}
	\caption{Generation of internal tides at Macquarie Ridge. (a) The heatmap illustrates 
	criticality in the region of major generation. And ellipses are representative of barotropic 
	current. (b) The diagnosed conversion rates for 'uniform' experiment. The boxes outline regions 
	used in further analysis. (c) Variability of conversion rates in the three regions identified 
	on the previous panel.}
	\label{C3.fig:gen}
\end{figure}

\begin{figure}
	\centering
	\includegraphics[scale = 
0.5]{/home/dmitry/Work/Research/thesis/FINALE/P3_ITS_GENERATION/figures/fig_4_stand_wave_2by2.png}
	\caption{Standing wave between Macquarie Ridge and Auckland Escarpments. (a) Distribution of 
	horizonal kinetic energy in 'uniform' simulation with overlaid energy flux of total mode-1 
	field. (b) Energy flux map of the westward component and the eastward component (c). Variation 
	in energy characteristics of mode-1 field between Macquarie Ridge and Aucklands Escarpment. The 
	integrated flux was obtained for a 	dashed transect in (c) and is given by arrows. Dots show 
	conversion rates spatially integrated over boxes given on (\fignm{C3.fig:gen}, b).}
	\label{C3.fig:stand_wave}
\end{figure}

\begin{figure}
	\centering
	\includegraphics[scale = 
	0.5]{/home/dmitry/Work/Research/thesis/FINALE/P3_ITS_GENERATION/figures/fig_5_2d_section_sill.png}
	\caption{Distribution of baroclinic fields during ebb and slack tide along transect crossing 
	Macquarie Ridge on \fignm{C3.fig:gen}, (b). (a-d) Baroclinic pressure anomaly with superposed 
	isopycnal displacements given by contour lines. Positive color is assigned to lifted interfaces 
	and negative - for the opposite. Stick lines show currents. And white line is distribution 
	of barotropic velocity. The tideward and leeward sides of Macquarie Ridge are identified by 
	vertical solid and dashed lines. On (e) it is shown time progression of 
	baroptropic-to-baroclinic conversion for both experiments and ridge sides. While (f) gives 
	distribution of period averaged conversion rate along the transect. The same is for (g) where 
	progression of mode-1 baroclinic pressure is shown by phases of total field and elemental 
	components. The phases are referenced to flood current across the sill.}
	\label{C3.fig:gen_2d}
\end{figure}

\begin{figure}
	\centering
	\includegraphics[scale = 
	0.5]{/home/dmitry/Work/Research/thesis/FINALE/P3_ITS_GENERATION/figures/fig_6_scatter_generation.png}
	\caption{Variation of the tideward conversion rates in relation to mode-1 energy transmitted 
	across the sill (a), to travel time across Solanders Trough (b), to knife-edge barrier 
	representative of Macquarie Ridge seeing from the deep ocean.}
	\label{C3.fig:gen_regr}
\end{figure}

\begin{figure}
	\centering
	\includegraphics[scale = 
	0.5]{/home/dmitry/Work/Research/thesis/FINALE/P3_ITS_GENERATION/figures/fig_8_along_beam.png}
	\caption{Along beam variation of across averaged energy characteristics. For all panels - solid 
	black line is mean value and thin gray lines for each particular realization. Purple lines show 
	mean deviation of realizations for mean. (a) Direction of energy flux. The dashed line is 
	produced by regression with second order polynomial. (b) Integrated flux carried by the beam. 
	Here the dashed line shows mean integrated flux with no account for relative angle with respect 
	to across beam surface. (c) Ratio between horizontal kinetic energy to available potential 
	energy. The dashed line is the ratio for a plane wave.}
	\label{C3.fig:beam_prms}
\end{figure}

\begin{figure}
	\centering
	\includegraphics[scale = 
	0.5]{/home/dmitry/Work/Research/thesis/FINALE/P3_ITS_GENERATION/figures/fig_9_APE_beam.png}
	\caption{Superposition of multiple waves.}
	\label{C3.fig:beam_dcmp}
\end{figure}

\begin{figure}
	\centering
	\includegraphics[scale = 
	0.5]{/home/dmitry/Work/Research/thesis/FINALE/P3_ITS_GENERATION/figures/fig_10_decomp_cumsum.png}
	\caption{Variation of the beam.}
	\label{C3.fig:beam_dcmp_cm}
\end{figure}

\begin{figure}
	\centering
	\includegraphics[scale = 
	0.5]{/home/dmitry/Work/Research/thesis/FINALE/P3_ITS_GENERATION/figures/fig_11_along_beam_inv.png}
	\caption{Variation of beam parameters.}
	\label{C3.fig:beam_alng_var}
\end{figure}

\bibliographystyle{apa}
\bibliography{/home/dmitry/Bibtex_lib/my_first_lib}

\end{document}